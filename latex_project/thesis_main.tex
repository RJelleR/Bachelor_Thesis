\documentclass[a4paper,11pt]{article}

\usepackage{times}
\usepackage[T1]{fontenc}
%\usepackage[xetex]{graphicx} % when you compile with xelatex
\usepackage[pdftex]{graphicx} % when you compile with pdflatex
\usepackage{epstopdf}
\usepackage{float}
\usepackage{enumerate}
\usepackage{boxedminipage}
\usepackage{url}
%\usepackage[dutch]{babel} % when you write in dutch

\usepackage{dcolumn}
\newcolumntype{d}[1]{D{.}{.}{#1}}

%%%%%%%%%%%% layout %%%%%%%%%%%%%%%%%%
\usepackage[margin=3cm]{geometry}
% you might enhance readability by increasing the distances between lines:
%\renewcommand{\baselinestretch}{1.1}

%%%%%%%%%%% mathematics %%%%%%%%%%%%%%%%%%
\usepackage{amsmath,amsfonts,amssymb}

%%%%%%%%%%%%% theorems %%%%%%%%%%%%%%%%
% adapt when you write Dutch

\usepackage{amsthm}

\theoremstyle{plain}
\newtheorem{theorem}{Theorem}
\newtheorem{corollary}{Corollary}
\newtheorem{proposition}{Proposition}
\newtheorem{lemma}{Lemma}
\newtheorem{assumption}{Assumption}

\theoremstyle{definition}
\newtheorem{definition}{Definition}
\newtheorem{example}{Example}
\newtheorem{remark}{Remark}
\newtheorem{notation}{Notation}

%%%%%%%%%%%%algorithms %%%%%%%%%%%%%%%%%%%
\usepackage{algorithm}
\usepackage{algorithmic}

%%%%%%%%% bibliography style %%%%%%%%%%%%%
\usepackage{natbib} % in case of classic bibtex
%\usepackage[natbib,style=authoryear,maxcitenames=2,
%           maxbibnames=99,hyperref,backend=biber]{biblatex}
%\addbibresource{bib/EORthesis.bib}

%%%%%%%%%%%%%%%% start document %%%%%%%%%%%
\begin{document}

%%%%%%%%%%%%%%% front page %%%%%%%%%%%%%%%%

\thispagestyle{empty}

\includegraphics[height=2cm]{figures/LogoSBE.png}

\vspace*{3cm}

\noindent
\rule{\textwidth}{0.8pt}
\begin{center}
{\huge\bf
\noindent
Deriving fractional moments using the Moment Generating Function
}
\end{center}

\vspace*{-8pt}
\noindent
\rule{\textwidth}{0.8pt}

\vspace*{2cm}

\begin{center}
{\LARGE\bf
Jelle Reisinger
}

{\Large
\vspace*{0.5cm}
(2780350)


\vspace*{2cm}

30 June 2025
}
\end{center}

\vspace*{2cm}

{\Large
\noindent
Bachelor Thesis Econometrics and Data Science
}

\vspace*{1cm}

{\Large
\noindent
Thesis commission:\\[0.3cm]
Prof. dr. xxxx (supervisor)\\[0.3cm]
Dr. yyyy (co-reader)
}


\newpage

%%%%%%%%%%%%%%%%%%%%%%% main contents %%%%%%%%%%%%%%%%%
\setcounter{page}{1}

\section*{Abstract}
The abstract should summarize the contents of the thesis.
It should be clear, descriptive, self-explanatory and not longer
than a third of a page. Please avoid using mathematical
formulas as much as possible.
Keywords might be given.

\bigskip\noindent
\textbf{Keywords:} Fractional moments, Moment Generating Function.

\section{Introduction}\label{s:intro}
Statistical moments, defined as the \(n\)-th moment of a random variable \(X\) via its probability density function, are essential tools for characterizing data and its distribution. The first and second order moments, the mean and variance, respectively provide key insights into the average value and dispersion of a random variable. Higher-order moments offer further information about the shape and symmetry of the distribution. A less commonly discussed class of moments, however, are the fractional moments. The latter are defined in precisely the same manner as the integer moments, but now with \(n \in \mathbb{R}\), or even \(n \in \mathbb{C}\). From this point onward, when referring to fractional moments, we denote the order by \(\alpha \in \mathbb{R}\) instead of \(n\). While these moments may not find as much usage in comparison with the integer moments of a distribution, they can be rather valuable in specific applications.
\newline
Fractional moments play a significant role in various fields, including finance, economics, and statistics. One notable application in statistics is their use in approximating integer moments, as described by \cite{inverardi2024}. In such cases, evaluating a 'nearby' existing fractional moment can provide an approximation of the otherwise undefined integer moment \cite{inverardi2024}.
\newline
As an illustration, one can consider the student-t distribution with \(\nu = 2\) degrees of freedom. This distribution only has central and raw moments of order \(k\), where \( 0 < k < \nu\). This implies that its second central moment \((k = 2)\) does not exist. One could however consider the central moment of order \(k\), where \(k = 1.95\), given that this order of the moment does in fact exist and interpret its value as the variance of the distribution. 
\newline
Fractional moments are also used in financial modelling, particularly in the context of Generalized Autoregressive Conditional Heteroskedasticity (GARCH) models. These models are commonly applied to time series data such as financial returns and capture the dynamic volatility that changes over time. The GARCH model achieves this by modelling the volatility based on the returns and variances of previous time periods. It is useful to note that in the context of financial returns, the variable of interest often follows a distribution with heavy tails. As a consequence, not all relevant moments of integer order may exist. In such cases, fractional moments may serve as a viable alternative.  \cite{hansen2024} propose a method for computing fractional absolute moments of the cumulative return, quantities that would otherwise be undefined using traditional integer-order techniques. 
\newline 
In addition, \cite{hansen2024} apply a Heterogenous Autoregressive Gamma Model (HARG) to model intraday stock data. The HARG model assumes the conditional distribution of the variable of interest, say \(X_t\) to be of a non-central Gamma distribution. This model is commonly used to model positive-valued time series data \cite{gourierroux2006}. In particular, the volatility of the random variable is of interest, which by definition is non-negative.
\newline
In this context, \(X_t\) is the realized variance. \cite{hansen2024} study the conditional moments of the variance process. Specifically, they consider moments of order \(\frac{1}{2}, \frac{3}{2}\) and 2 which correspond to the standard deviation, skewness and kurtosis respectively. Furthermore, the moment of order \(-\frac{1}{2}\), the inverse volatility has also been computed. In this context, this moment serves as an estimate of the Sharpe ratio, which is an index to measure the performance of some investment compared to a risk-free asset \cite{sharpe1994}. These fractional moments offer valuable insights into financial return dynamics and are highly relevant for modelling and risk assessment. 
\newline
In a related application, \cite{Mikosc2013} apply fractional moments to financial time series data, focusing on the stationary solution of a stochastic recurrence equation - a recursive formula that describes the evolution of a time series. This equation generalizes common time series models in econometrics such as the ARCH and GARCH and is thus of great interest in the context of financial econometrics. Notably, the GARCH(1,1) model can be represented within this one-dimensional stochastic recurrence equation framework \cite{Mikosc2013}. 
\newline
Similar to the case study by \cite{hansen2024}, conditional moments of fractional order are of great interest, as is often the case in the context of (G)ARCH models. The latter can again be explained by the fact that fractional moments offer a valuable alternative for characterizing the distributional properties of financial time series.
\newline
\cite{gyzl2013} also introduce the use of Fractional moments in risk-models, specifically insurance models. In such models, often the probability density function of total ruin, the event where an insurance company's capital becomes negative, is unknown or difficult to compute. The method of maximum entropy has been developed as a way to approximate these unknown density functions. This method uses fractional moments as input, as they have been proven to be able to characterize its distribution \cite{lin1992}. Compared to traditional techniques, such as inverse Laplace transforms, the maximum entropy approach requires fewer moments to achieve a reliable approximation, making it more computationally efficient and practical for real-world applications. 
\newline
In prior research by \cite{DAmico2002}  the same approach of using fractional moments as input for maximum entropy methods to characterize unknown distributions has been employed. One notable application is in option pricing in A Black-Scholes model. Making use of fractional moments, numerical approximations of theoretical entropies have been obtained. Remarkably, using only two fractional moments, the method achieved approximations with three-digit accuracy.
\newline
 This approach outperformed traditional maximum entropy methods that rely on integer moments, both in terms of accuracy and computational efficiency. Moreover, the use of fractional moments helps to avoid numerical optimization issues that commonly arise when minimizing functions based on integer-moment constraints. Therefore, the results obtained by \cite{DAmico2002} seem to correspond with the aforementioned results obtained by \cite{gyzl2013}. Namely, the implementation of fractional moments in the context of maximum entropy methods has great advantages when it comes to numerical stability and efficiency as compared to more traditional methods. 
\newline
Beyond finance and risk modelling, fractional moments have found valuable applications in fields such as engineering and healthcare. As previously mentioned, \cite{gyzl2024} used fractional moments combined with the maximum entropy method to predict total ruin. This approach has been extended to estimating lifetime distributions, a general topic of interest in survival analysis \cite{clark2003}. Both probability density functions and survival functions were estimated using this technique. Once again, the prediction errors remained within three-digit accuracy, further supporting the method’s consistency and precision across different fields of research.
\newline
Other examples in the field of engineering include optimizing signal processing and control systems as well as studying the response characteristics of random vibration systems. \cite{wang2025} have shown that when using the concept of fractional moments for the latter, accuracy and stability is higher compared to traditional methods, such as Taylor expansions. Furthermore, in terms of simplicity and efficiency, the method of fractional moments is advantageous, as its computation steps are straightforward and avoid convergence issues, significantly reducing the resources required for computation. Implementing the usage of fractional moments has allowed \cite{wang2025} to obtain both analytical, as well as numerical solutions to problems within their research field, which again proves its viability. 
\newline
Another noteworthy application of fractional moments is in the identification of distributions in complex, non-linear systems. \cite{dimatteo2014} demonstrate how complex fractional moments can be used to solve differential equations such as the Kolmogorov or Fokker-Planck equation, which arise in the context of continuous-time Markov processes.
By transforming the system into a more agreeable form using Mellin transforms, \cite{dimatteo2014} show that solutions can be efficiently computed and later recovered using inverse transformations. A key advantage of this method is that it preserves the entire support of the probability distribution, which traditional integer moment methods fail to achieve.
\newline
The remainder of this paper is structured as follows.
Section \ref{s:methodology} discusses existing methods for computing fractional moments, outlining their respective advantages and disadvantages. This section concludes with an introduction to the novel approach proposed in this paper, the required techniques, and the limitations to be mindful of. Section \ref{s:calculus} lays the mathematical foundation for this new method and provides a brief historical overview of the development of fractional derivatives. In section \ref{s: moments}, some key definitions and properties of statistical moments and the moment generating function will be highlighted. What is more, the theory of fractional derivatives from section \ref{s:calculus} will be incorporated in order to extend functionality of the moment generating function. Relevant properties are revisited in this new context, and potential analytical errors of the method are addressed. In section \ref{s:simulation},  simulation results will be analysed, which further clarify the numerical stability and potential errors of computing fractional moments via the moment generating function. Finally, in section 6, the practical implementation of this method in a financial setting will be implemented. The results will be compared against those obtained using traditional methods, with a focus on the implications of potential errors in real-world applications.


\section{General Guidelines}\label{s:general}
\section{Fractional Calculus}\label{s:calculus}
In order to obtain the expression as mentioned in section \ref{s:intro}, some advanced tools are required. We can find this in the field of fractional calculus.
We define the following:
\begin{definition}
    Let \(D\) be the differential operator, such that \(D f(x) = \frac{d}{dx} f(x)\). Then the fractional derivative of order \(\alpha\) is defined as \(D^{\alpha} f(x) = \frac{d^{\alpha}}{dx^{\alpha}} f(x)\).
\end{definition}
In this definition, \(\alpha\) can be any real number. When taking regular derivatives, \(\alpha \in \mathbb{N}\). In our case, we are interested in instances where  \(\alpha \in \mathbb{Q_{\geq 0}}\).

It is also possible to study derivatives of negative order, which can be used to obtain moments of negative order of a function. A derivative of negative order is simply an integral of positive order. This is defined as follows:
\begin{definition}
    Let \(J\) be the integral operator, such that \(I f(x) = \int f(x) dx\). Then the fractional integral of order \(\alpha\) is defined as \(I^{\alpha} f(x) = \int f(x) dx^{\alpha}\).
\end{definition}

A lot of different definition have been used to compute a fractional derivative. In this paper, we will focus on the following fractional derivatives:
\begin{definition}
    The Riemann-Louville fractional derivative of order \(\alpha\) is defined as:
    \begin{equation}
        D^{\alpha} f(x) =  \frac{d^{n}}{dx^{n}} D_{x}^{-(n - \alpha)} f(x) = \frac{d^{n}}{dx^{n}} I_{x}^{n - \alpha} f(x) = \frac{1}{\Gamma(\alpha)}  \int_{0}^{x} (x-t)^{\alpha-1} f(t) dt
    \end{equation}
    Where \(n = \lceil \alpha \rceil\), the ceiling function and \(\Gamma(.)\) is the Gamma function, see section \ref{s:appendices}.
   
\end{definition}
A modification of the Riemann-Louville derivative is the Caputo derivative, which is defined as follows:
\begin{definition}
    The Caputo fractional derivative of order \(\alpha\) is defined as:
    \begin{equation}
        D^{\alpha} f(x) = \frac{1}{\Gamma(n-\alpha)}  \int_{0}^{x} (x-t)^{n-\alpha-1} f(t) dt
    \end{equation}
    Where, again \(n = \lceil \alpha \rceil\), the ceiling function and \(\Gamma(.)\) is the Gamma function.
    
\end{definition}

Lastly, we will define the Riesz derivative, which is defined as follows:
\begin{definition}
    The Riesz fractional derivative of order \(\alpha\) is defined as:
    \begin{equation}
        D^{\alpha} f(x) = -\int_{-\infty}^{\infty} |\xi|^{\alpha} \hat{f}(\xi) e^{2\pi i  x \xi} d\xi
    \end{equation}
    Where \(\hat{f}(\xi)\) is the Fourier transform of \(f(x)\).
\end{definition}


\subsection{Formatting the Front Page}
The front page just contains
\begin{itemize}
\setlength\itemsep{-1mm}
\item
Title of the thesis.
\item
Author (your name and student number).
\item
Date.
\item
References or logo(s) of
\begin{itemize}
\item
VU University.
\item
School of Business and Economics.
\item
Department of Econometrics and Operations Research.
\end{itemize}
\item
Master Thesis.
\item
Thesis Committee and its members.
\end{itemize}


\section{Mathematics}
\subsection{Mathematical Expressions in Text and in Displays}
Display only the most important equations, and number only the
displayed equations that are explicitly referenced in the text.
To conserve space, simple mathematical expressions such as
$\bar Y = n^{-1} \sum_{i=1}^n Y_i$ may be incorporated into the text.
Mathematical expressions that are more complicated or that must be
referenced later should be displayed, as in
\[
s^2 = \frac 1 {n-1} \sum_{i=1}^n (Y_i - \bar Y)^2.
\]

If a display is referenced in the text, then enclose the equation number
in parentheses and place it flush with the right-hand margin of the
column. This is automatically obtained by the \texttt{equation} environment
accompanied by the \verb+\label+ command.
For example, the quadratic equation has the general form

\begin{equation} \label{eq:quadratic}
ax^2 + bx + c = 0, \mbox{ where } a \ne 0.
\end{equation}

In the text, each reference to an equation number should also be enclosed in
parentheses by using \verb+\eqref{<labelgiven>}+.
For example, the solution to \eqref{eq:quadratic} is given
in \eqref{eq:quadraticsol} in Appendix \ref{app:quadratic}.

If the equation is at the end of a sentence, then you should end the equation
with a period. If the sentence in question continues beyond the equation,
then you should end the equation with the appropriate
punctuation---that is, a comma, semicolon, or no punctuation mark.

\subsection{Symbols, Commands, Environments, Etc}
See

\url{https://en.wikibooks.org/wiki/LaTeX/Mathematics}

\noindent
and

\url{https://en.wikibooks.org/wiki/LaTeX/Advanced_Mathematics}.


\subsection{Definitions, Theorems, Etc.}
Definitions, theorems, propositions, etc. should be formatted
using the \texttt{amsthm} package.
Number these items separately and sequentially. Examples are given below

\begin{definition}
In colloquial New Zealand English, the term \textit{dopey mongrel} is used
to refer to someone who has exhibited less than stellar intelligence.
\end{definition}

\begin{theorem}\label{t:delete}
If a proceedings editor from New Zealand accidentally deletes his draft of
the author kit shortly after completing it, he would be considered to be a dopey mongrel.
\end{theorem}
\begin{proof}
The proof follows by the principle of contradiction.
Suppose the editor is not a dopey mongrel, then he is smart enough to save
the author kit.
\end{proof}

\begin{corollary}
One of the proceedings editors is a dopey mongrel.
\end{corollary}
\begin{proof}
This follows immediately from Theorem \ref{t:delete}.
\end{proof}

\section{Figures and Tables}
\label{sec:graphics}
Figures and tables should be centered within the text and should not
extend beyond the right and left margins of the paper.
Figures and tables can make use of color.
However, try to select colors that can be differentiated when printing
in black and white in consideration of vast majority of people
using such printers.
Figures and tables are numbered sequentially, but separately,
using arabic numerals.

\subsection{Tables}
Each table should appear in the document after the paragraph in which the
table is first referenced. However, if the table is getting split across
pages, it is okay to include it after a few paragraphs from its first reference.
One-line captions are centered, while multiline captions are left justified.
The captions appear \textit{above} the table. See Tables \ref{tab:first} and
\ref{tab:second} for examples.

\begin{table}[H]
\caption{Table captions appear above the table, and if they are longer
than one line they are left justified. Captions are written using normal
sentences with full punctuation. It is fine to have multiple-sentence
captions that help to explain the table.}
\label{tab:first}
\medskip
\centering
\begin{tabular}{rll}
\hline
Creature & IQ & Diet\\ \hline
dog & 70 & anything\\
cat & 75 & almost nothing\\
human & 60 & ice cream \\
dolphin & 120 & fish fillet\\
\hline
\end{tabular}
\end{table}

\begin{table}[H]
\centering
\caption{Counting in Maori.}
\label{tab:second}
\medskip
\begin{tabular}{r|l}
English & Maori \\ \hline
one & tahi \\
two & rua \\
three & toru \\
four & wha \\
\hline
\end{tabular}
\end{table}

\begin{table}[H]
  \centering
  \caption{Alternative table aligning columns}
  \label{tab:alt}
  \medskip
  \begin{tabular}{r|d3d5}
       & \multicolumn{1}{c}{2012} & \multicolumn{1}{c}{2013} \\
    \hline
    Mean & 1.23 & 1.23456 \\
    Variance & 13454.4 & 3435.456 \\
    \hline
  \end{tabular}

  \medskip
  \parbox{100mm}{\small Note: Alternatively, place a longer explanation
  as a note at the bottom of the table, allowing the caption to be more
  concise. This table uses explicit alignment of the numbers in the
  columns, using either 3 or 5 decimals.}
\end{table}


\subsection{Figures}
Each figure should appear in the document after the paragraph in which
the figure is first referenced. One-line captions are centered,
while multiline captions are left justified. Figure captions appear
below the figure. 
To include figures you use package \texttt{graphicx}, and in the document the
command

\verb+\includegraphics{<graphs/graphicfilename>}+.

The graphicfilename usually does not have to include an extension;
PdfLaTeX would search for extensions it recognises. In general, it is
advisable to keep your graphs in a separate directory, to avoid clogging
up your LaTeX directory.

If the graph is scaled correctly, one should be able to use a

\verb+\includegraphics[width=\textwidth]{<graphs/graphicfilename>}+

to ensure that the graph fills the width of the paper.




\noindent



\subsection{References to Tables and Figures}
References to tables and figures identified by number are capitalized.
For example, ``We see in Table \ref{tab:second} that...'' and
``We see in the previous table that...'' are both correct.
Be sure to use the \verb+\label+
command within the figure or table environment and refer to the
associated figure or table using \verb+Table \ref{<labelgiven>}+.
Please do not use hard coded figure/table numbers.
This is error prone.

\subsection{Graphics Formats}
As graphics files in your document you use \texttt{.jpg}, \texttt{.png}, 
\texttt{.pdf}, or \texttt{.eps} files.
But there are tools to convert these formats into one another.
The main difference between the formats is how they store the images and how
well suited they are for specific graphics. In general we can
choose between bitmap and vector graphics.
Bitmap graphics are well suited for photographies (jpg is very common here)
or for screenshots (png is a lossless encoding (in contrast to jpg),
and is thus better suited for all those cases where you have sharp edges in your graphics).
Vector graphics are the encoding to be chosen for all kinds of drawings
(diagrams, charts, ...). In contrast to bitmap formats, they can be scaled
to any size without any loss of sharpness. This makes it possible to
read such graphics even if two pages are printed on one sheet of paper,
or if the documents are read electronically.

So what to choose for your Latex document? As a rule of thumb you should
always prefer pdf and eps. In general these two encodings can contain both,
bitmap and vector graphics. But there is no need (and no use) to convert
your bitmaps to any of these.

You include figures via the \verb+\includegraphics+ command.
You must use the \texttt{pdflatex} or \texttt{xelatex} command to generate
your pdf file, as was done with this file.

\section{Algorithms}
Typeset algorithms by using the \texttt{algorithmic} environment of the
\texttt{algorithm} package.
The command \verb+\begin{algorithmic}+ can be given the optional argument of a
positive integer, which if given will cause line numbering to occur at multiples
of that integer. E.g. \verb+\begin{algorithmic}[5]+ will enter the algorithmic
environment and number every fifth line. Below is an example of typesetting a
basic algorithm (remember to add the

\verb+\usepackage{algorithm,algorithmic}+

\noindent
statement to your document preamble). The pseudocode is centered by using the
\texttt{minipage} environment.

{\small
\begin{verbatim}
\begin{center}
\begin{minipage}{10cm}
\begin{algorithm}[H]
\caption{Polar Method for Normal Random Numbers}
\begin{algorithmic}[1]
\REPEAT
   \STATE $U_1\sim {\sf U}(0,1)$ \COMMENT{Generate uniform on $(0,1)$}
   \STATE $U_2\sim {\sf U}(0,1)$
   \STATE $V_1 \gets 2U_1 -1$ \COMMENT{Uniform on $(-1,1)$}
   \STATE $V_2 \gets 2U_2 -1$
   \STATE $W \gets V_1^2 + V_2^2$
\UNTIL {$W < 1$}
\STATE \textbf{return} $V_1\,\sqrt{(-2\,\ln(W)/W}$
\end{algorithmic}
\end{algorithm}
\end{minipage}
\end{center}
\end{verbatim}
}

\noindent
This produces

\begin{center}
\begin{minipage}{10cm}
\begin{algorithm}[H]
\caption{Polar Method for Normal Random Numbers}
\begin{algorithmic}[1]
\REPEAT
   \STATE $U_1\sim {\sf U}(0,1)$ \COMMENT{Generate uniform on $(0,1)$}
   \STATE $U_2\sim {\sf U}(0,1)$
   \STATE $V_1 \gets 2U_1 -1$ \COMMENT{Uniform on $(-1,1)$}
   \STATE $V_2 \gets 2U_2 -1$
   \STATE $W \gets V_1^2 + V_2^2$
\UNTIL {$W < 1$}
\STATE \textbf{return} $V_1\,\sqrt{(-2\,\ln(W)/W}$
\end{algorithmic}
\end{algorithm}
\end{minipage}
\end{center}

\section{Bibliography Management}
Managing your bibliography requires two specifications:
the citation style in your document,
and the style of the reference list at the end of the thesis.

\subsection{Citing a Reference}
To cite a reference in the text, use the author-date method. Thus,
\citet{ross06} could also be cited parenthetically \citep{ross06}.
For a work with three or more authors, use an abbreviated form.
For example, a work by Evans, Keith and Kroese would be cited in
one of the following ways:
\citet{evkekro07} or \citep{evkekro07}.

Parenthetical citations are enclosed in parentheses $(~)$, not square brackets $[~]$.
The items in a series of such citations are usually separated by commas.
If an item in the series of parenthetical citations contains punctuation 
because (for example) it refers to a work with three or more coauthors, 
then all items should be separated by semicolons.

The following is a list of correct forms of citations:
\begin{itemize}
\setlength\itemsep{-1mm}
\item Brown and Edwards (1993),
\item (Brown and Edwards 1993),
\item (Brown and Edwards, 1993),
\item Brown and Edwards (1993), Smith (1997),
\item (Brown and Edwards 1993; Smith 1997; Brown et al. 1997).
\item (Brown and Edwards, 1993; Smith, 1997; Brown et al., 1997).
\end{itemize}

\subsection{List of References}
Place the list of references after the appendices.
The section heading is \textbf{References}, and it is not numbered.
List only references that are cited in the text.
Arrange the references in alphabetical order (chronologically for a
particular author or group of authors); do not number the references.
Give complete references without abbreviations.
To identify multiple references by the same authors and year, append a
lower case letter to the year of publication; for example, 1984a and 1984b.

Use hanging indentation to distinguish individual entries.
Do not insert additional space between references.

The bibliographic style for a journal article is: \\
$<$Surname of first author$>$, $<$First author initials$>$,
$<$Initials and surnames of other authors$>$ ($<$year$>$).
$<$Article title$>$. $<${\em Journal Name in
Headline Italics}$>$ $<$Volume number$>$, $<$page numbers$>$.

The article title may come between quotes but this is not required.
The format for other types of reference can be inferred from the
examples in the references, which include:
\begin{itemize}
\setlength\itemsep{-1mm}
\item a technical report \citep{chi89},
\item a proceedings article \citep{evkekro07},
\item a journal article \citep{alon},
\item a book by 2 authors \citep{asmus07},
\item a chapter in a book \citep{asrub1},
\item an unpublished thesis or dissertation \citep{garvels00},
\item a document available on the web \citep{sudoku}.
\end{itemize}

\subsection{BibTeX}
These formats are obtained by creating a BibTeX database for all your references.
A BibTeX database is stored as a \texttt{.bib} file.
It is a plain text file, and so can be viewed and edited easily.
One benefit of using BibTeX is that bibliography formatting and
referencing can be greatly simplified: the correct citation and
reference list style is automatically created.
How the BibTeX items are entered in the \texttt{bib} file will be
explained in section \ref{ss:bibtex}.

Once you have created the
\texttt{bib} file with your references it needs to be included
in your document.

\subsection{Biblatex Package}
Advise is to use the \texttt{biblatex} package. It is a reimplementation of the
classic BibTeX providing modern bibliographic facilities such as
advanced name disambiguation, smart crossref data inheritance,
configurable sorting schemes, dynamic datasource modification, etc.
The prerequisite is that you need to use Biber to process your \texttt{bib}
files. Older TeX distributions do not have Biber included, you need to check this
or you find out when you cannot run your \texttt{bib} files using the
\texttt{biblatex} package.
In that case, see next section for the classic BibTeX implementation.

\begin{itemize}
\item
In the preamble you put

\begin{verbatim}
\usepackage[natbib,
            style=authoryear,
            maxcitenames=2,
            maxbibnames=99,
            hyperref,
            backend=biber]{biblatex}
\addbibresource{bib/<bibfilename>.bib}
\end{verbatim}

\item
At the end of your document, where the reference list is supposed to come you
put

\verb+\printbibliography+

\item
In your document, wherever you need
to cite a reference you put \verb+\citet{<bibentrykey>}+ (for the author name(s)
followed by the year in parentheses), or
\verb+\citep{<bibentrykey>}+
to get the entire citation in parentheses.
\end{itemize}

\subsection{Classic BibTeX}
In case Biber does not work in your TeX distribution.
Use the \texttt{natbib} bibliography style with \texttt{chicago} citation style.

\begin{itemize}
\item
In the preamble you put

\verb+\usepackage{natbib}+.

\item
At the end of your document, where the reference list is supposed to come you
put

\begin{verbatim}
\bibliographystyle{chicago}
\bibliography{bib/<bibfilename>.bib}
\end{verbatim}

\item
Citations are \verb+\citet{<bibentrykey>}+ (for the author name(s)
followed by the year in parentheses), or
\verb+\citep{<bibentrykey>}+
to get the entire citation in parentheses.
\end{itemize}

\subsection{BibTeX Entries}\label{ss:bibtex}
See the \texttt{EORthesis.bib} file available in Canvas,
or use the templates that you can find, e.g.,

\url{https://en.wikibooks.org/wiki/LaTeX/Bibliography_Management}

\noindent
and

\url{https://en.wikipedia.org/wiki/BibTeX}.

\subsection{Compilation}
You need to obtain the final pdf version of document with correct citations and reference
list.

\begin{enumerate}[(a).]
\item
In case you use Biblatex, you run consecutively
\texttt{pdflatex, biber, pdflatex}. In stead of
\texttt{pdflatex}, you might also run \texttt{xelatex}.
\item
In case you use classic BibTeX, you run consecutively
\texttt{pdflatex, bibtex, pdflatex, pdflatex}. In stead of
\texttt{pdflatex}, you might also run \texttt{xelatex}.
\end{enumerate}

\subsection{Example}
In \citet{asmus07} we see that they mention \citep{blitzstein10},
and after some more comments of \citet{asrub1}, we know that
\citet{evkekro07} have published the same results
as we can find in \citep{alon,garvels00,ross06},
however in a more general context.

\section{Conclusion}\label{s:con}
The conclusion summarizes the results of your research.
Also it gives possible directions of further research.

\section*{Acknowledgements}
Place the acknowledgments section, if needed, after the main text, 
but before any appendices and the references. The section heading is not numbered.

\newpage
\appendix

\section{(Appendix): Relevant functions and Identities}\label{s:appendices}
We define the (Euler-)Gamma function as follows:
\begin{definition}\label{d: eg}
    for \(\Re(z) > 0\), we have the following: \(\Gamma(z) = \int_{0}^{\infty} t^{z-1} e^{-t} dt\)
\end{definition}

The Gamma function can be seen as an extension of the factorial function, for non-integers. This function is defined for complex numbers and all there subsets (so also real numbers), as long as the condition above holds. For positive integers values \(z\), we have the following identity: \(\Gamma(z) = (z - 1)!\)
Other important identities, not necessarily for \(z\) an integer, are: 
\begin{itemize}
    \item \(\Gamma(z + 1) = z \Gamma(z)\)
    \item \(\Gamma(2) = \Gamma(1) = 1\)
    \item \(\Gamma(\frac{1}{2}) = \sqrt{\pi}\)
\end{itemize}

\begin{definition}\label{d: ff}
    The falling factorial is defined as follows: \((x)_n = \prod_{k = 0}^{n - 1} (x - k)\), which is a polynomial
\end{definition}
\begin{definition}
    For \(0 \leq k \leq n\), the Binomial Coefficient is defined as follows: \(\binom{n}{k}\), where \(n, k \in \mathbb{N}\).
\end{definition}
We can derive the following factorial identity, which is convenient to work with analytically: \(\binom{n}{k} = \frac{n!}{k! (n - k)!}\).
For numerically computing expressions containing the Binomial Coefficient, the following identity is computationally more efficient: \(\binom{n}{k} = \frac{(n)_k}{k!}\). With \((n)_k\) as in \ref{d: ff}.
Since we have established in \ref{d: eg} that \(\Gamma(z) = (z - 1)!\), we can rewrite our factorial identify to:
\[\binom{n}{k} = \frac{\Gamma(n + 1)}{\Gamma(k + 1) \cdot \Gamma( n - k  + 1)} = \frac{n}{k} \cdot\frac{\Gamma(n)}{\Gamma(k) \cdot \Gamma(n - k + 1)}\]

\begin{definition}\label{d: binomial}
    The binomial series is a generalization of the binomial formula, namely:
    \[(1 + x)^\alpha = \sum_{k = 0}^{\infty}\binom{\alpha}{k} x^k\]
\end{definition}

\begin{definition}
    Vandermonde's identity: for non-negative integers, \(k, l, m, n\), we have that \[\sum_{k = 0}^{l} \binom{m}{k} \cdot \binom{n}{l - k} = \binom{m + n}{l}\]
\end{definition}
A modification on the latter identity has been called the Chu-Vandermonde identity. This is the same identity, but it his been proven that the identities still hold for complex values \(m, n\) as long as \(l\) is a positive integer (\cite{askey75}).

For the interchange-ability of derivatives and integrals and sums, we can apply the following two theorems:

\begin{theorem}
    Leibnitz's Rule:
    Let \(f(x, \theta), a(\theta), b(\theta)\) be differentiable with respect to \(\theta\), then we have that:
    \[ \frac{d}{d\theta} \int_{a(\theta)}^{b(\theta)} f(x, \theta) dx = f(b(\theta), \theta) \frac{d b(\theta)}{d\theta} - f(a(\theta), \theta) \frac{d a(\theta)}{d\theta} + \int_{a(\theta)}^{b(\theta)} \frac{\partial f(x, \theta) }{\partial \theta} dx.\] For the special case, where \(a(\theta), b(\theta)\) are constant we have that: 
    \[\frac{d}{d\theta} \int_{a}^{b} f(x, \theta) dx =  \int_{a}^{b} \frac{ \partial f(x, \theta)}{ \partial d\theta}.\]
\end{theorem}

For the interchange-ability of derivatives and summations, the following theorem has been given by \cite{casella2002}:
\begin{theorem}
    Suppose that the series \(\sum_{x = 0}^{\infty} h(\theta, x)\) converges for all \(\theta\) in an interval \((a, b)\) of real numbers and 
    
    \begin{enumerate}[(i)]
        \item \(\frac{\partial h(\theta, x)}{\partial \theta}\) is continuous for all \(x\)
        \item \(\sum_{x = 0}^{\infty} \frac{\partial h(\theta, x)}{\partial \theta}\) converges uniformly on every closed bounded subinterval of \((a, b)\)
    \end{enumerate}
    Then:
    \[
    \frac{d}{d \theta} \left( \sum_{x = 0}^{\infty} h(\theta, x) \right) = \sum_{x = 0}^{\infty} \frac{\partial h(\theta, x)}{\partial \theta}
    \]
    
   
\end{theorem}


\newpage
\section{(Appendix): Proofs}\label{s:app_B}

\subsection{Proofs section \ref{s:calculus}}

\begin{proof}
    Proof of \ref{p: calculus}
    \begin{enumerate}[(i)]
        \item We will proof for the Riemann-Liouville derivative, the proof for the Caputo-Fabrizio derivative is very similar and the Grünwald-Letnikov derivative is a direct consequence of the linearity of the sum.
        \[ D^{\alpha} (a f(x) + b g(x)) = \frac{d^n}{dx^n}\frac{1}{\Gamma(n - \alpha)} \int_{0}^{x} (x - t)^{n - \alpha - 1} (a f(t) + b g(t)) dt \]
     
        \[= \frac{d^n}{dx^n}\left(\frac{a}{\Gamma(n - \alpha)} \int_{0}^{x} (x - t)^{n - \alpha - 1}  f(t)dt + \frac{b}{\Gamma(n - \alpha)} \int_{0}^{x} (x - t)^{n - \alpha - 1} g(t)dt\right) \] Where we simply split the integral and put the constants in front.
        \[= \frac{d^n}{dx^n}\frac{a}{\Gamma(n - \alpha)} \int_{0}^{x} (x - t)^{n - \alpha - 1}  f(t)dt + \frac{d^n}{dx^n} \frac{b}{\Gamma(n - \alpha)} \int_{0}^{x} (x - t)^{n - \alpha - 1} g(t)dt \] As the regular derivative operator is linear.
        \[ = a D^{\alpha} f(x) + b D^{\alpha} g(x) \]
        \item Intuitively, this makes perfect sense, as the 0-th derivative is just no derivative, so just the function \(f(x)\). But for these derivatives, a little bit more effort is required to prove this rather obvious fact.
        \newline 
        For the Grünwald-Letnikov derivative we get: \[D^0 f(x) = \lim_{h \to 0} \frac{1}{h^0} \sum_{k=0}^\infty (-1)^k \binom{0}{k} f(x - k h)
        = \lim_{h \to 0} \frac{1}{1} \sum_{k=0}^\infty (-1)^k \frac{0!}{k!(0- k)!} f(x - k h).\] The factorial identity of the binomial coefficient only holds for \(0 \leq k \leq \alpha\). Since \(\alpha = 0\) and k is always a positive integer lesser or equal to \(\alpha, k = 0\). Thus, we get:
        \[ = \lim_{h \to 0} \sum_{k=0}^\infty (-1)^0 \frac{0!}{0!(0- 0)!} f(x - 0 h) = \lim_{h \to 0} f(x - 0 h) = f(x).\]
        \newline
        For the Caputo-Fabrizio derivative, we obtain the following:
        \[ D^{0} f(x) = \frac{1}{1 - 0}  \int_{0}^{x} \exp\left(\frac{0}{1 -0}(x-t)\right) f'(t) dt = \int_{0}^{x}f'(t) dt = f(x).\]
        \newline
        Finally, for the Riemann-Liouville derivative, we can simply make use of \autoref{d: differintegral} and \autoref{r: integer} to note that in this context \(\alpha = 0\) is included in the natural integers. So \(D^\alpha = \frac{d^\alpha}{dx^\alpha} f(x) = {d^0}{dx^0} f(x) = f(x)\) by the first fundamental theorem of calculus.

        \item The proof for the Riemann-Liouville derivative is given by \cite{koning15}. And the proof for the Caputo-Fabrizio derivative is given by \cite{losada15}.For the Grünwald-Letnikov derivative, we get:
        \[ D^\alpha(D^\beta f(x)) = \lim_{h \to 0} \frac{1}{h^\alpha} \sum_{k=0}^\infty (-1)^k \binom{\alpha}{k}(  \frac{1}{h^\beta} \sum_{l=0}^\infty (-1)^l \binom{\beta}{l} f(x - l h - kh))\]
        \[= \lim_{h \to 0} \frac{1}{h^{\alpha + \beta}} \sum_{k=0}^\infty (-1)^k \binom{\alpha}{k} \sum_{l=0}^\infty (-1)^l \binom{\beta}{l} f(x - (k + l)h).\] We substitute \(m = k + l\) to deal with the dubble sums: 
        \[ \lim_{h \to 0} \frac{1}{h^{\alpha + \beta}} \sum_{m=0}^\infty f(x - mh)  \sum_{k=0}^m (-1)^k (-1)^{ m - k} \binom{\alpha}{k} \binom{\beta}{m - k}\] Now we make use of an identify from \autoref{s:appendices} to obtain:
        \[ = \lim_{h \to 0} \frac{1}{h^{\alpha + \beta}} \sum_{m=0}^\infty (-1)^m \binom{\alpha + \beta}{m} f(x - mh) = D^{\alpha + \beta} f(x).\]
        It can be shown in an exactly similar way that the latter expression is equal to \(D^\beta(D^\alpha f(x))\).

        
    \end{enumerate}
\end{proof}

\subsection{Proofs section \ref{s: moments}}

\begin{proof}
    Proof of proposition \ref{p: moments_1}
     
    \begin{enumerate}[(i)]
        \item If \(X\) is a discrete random variable, we get: \[\mathbb{E}[X^n] = \sum_{i} x_i^n f_X(x_i) = \infty, \mathbb{E}[X^k] = \sum_{i} x_i^k f_X(x_i)\]
        \[ = \sum_{i} x_i^n \cdot x^{k - n} f_X(x_i) \geq \sum_{i} x_i^n f_X(x_i) = \infty, \text{ as } k \geq n\]
        We can prove this for continuous random variables in a similar manner.
        \item This simply follows from the previous proposition, as the current proposition is just the contrapositive statement of the previous proposition.
    \end{enumerate}
\end{proof}

\begin{proof}
    Proof of Proposition \ref{p: moments}
    \begin{enumerate}[(i)]
        \item This is trivial. For \(X\) a continuous random variable, we get: 
        \[ M_X^{(0)}(t) = \int_{-\infty}^{\infty} x^0 e^{t x } f_X(x) dx = \int_{-\infty}^{\infty} 1 e^{0 x } f_X(x) dx = \int_{-\infty}^{\infty} f_X(x) dx.\] Assuming that \(f(x)\) is a PDF, this integrates to 1 by definition. If this integral is not equal to 1, this implies that \(f(x)\) is not a PDF. The proof for the discrete case is the same but with a summation instead of an integral sign.
        \item \[M_{\mu + \sigma X}(t) = \mathbb{E}[e^{(\mu + \sigma X)t}] = \mathbb{E}[e^{\mu t} \cdot e^{\sigma X t}].\] Since this is the expectation of \(x\), every term that is not dependent on \(x\) can be taken out of the summation:
        \[ = e^{\mu t} \cdot \mathbb{E}[e^{\sigma X t}] = e^{\mu t} \cdot M_{ X}(\sigma t)\]
        \item 
        \[M_{X+Y}(t) = \mathbb{E}[e^{(X + Y)t}] = \int_{-\infty}^{\infty} \int_{-\infty}^{\infty} e^{(x + y)t} f_{X, Y}(x, y) dx dy\] 
        \[= \int_{-\infty}^{\infty} \int_{-\infty}^{\infty} e^{xt} \cdot e^{yt} f_{X, Y}(x, y) dx dy.\] 
        \(f_{X, Y}(x, y)\) is the joint pdf for \(X, Y\). But we know that the latter is equal to \(f_X(x) \cdot f_Y(y)\), if \(X, Y\) are independent. Thus we get:
        \[ = \left(\int_{-\infty}^{\infty}  e^{xt} f_X(x) dx\right) \left(\int_{-\infty}^{\infty}  e^{yt} f_Y(y) dy\right) = M_X(t) \cdot M_Y(t).\] 
    \end{enumerate}
\end{proof}

\begin{proof}
    Proof of Theorem \ref{t: negative}:
Suppose for the moment that \( X \) is a positive random variable. Since \( x \cdot f_X(x) \) is integrable for \( x > 0 \), we have:

\[
    \mathbb{E}(X) = \int_0^\infty x \, dF(x) = \int_0^\infty \int_0^\infty e^{tx} \, dt \, dF(x).
\]

We can interchange the order of integration as follows:

\[
    \mathbb{E}(X) = \int_0^\infty e^{tx} \, dF(x) \, dt = \int_0^\infty M_X(-t) \, dt.
\]

The interchange of the order of integration is subject to \( \mathbb{E}(e^{-tX}) \) being integrable from \( t = 0 \) to \( t = \infty \).

Finally, by substituting \( X^{-1} \) for \( X \), we find:

\[
    \mathbb{E}(X^{-1}) = \int_0^\infty M_X^{-1}(-t) \, dt.
\]

There are two natural ways to generalize (1) to \( \mathbb{E}(X^{-1}) \); one way gives:

\[
    \mathbb{E}(X^{-n}) = \int_0^\infty \int_0^\infty \cdots \int_0^\infty M_X(-t_n) \, dt_n \cdots dt_2 dt_1, \tag{2}
\]

while the second way gives:

\[
    \mathbb{E}(X^{-n}) = \frac{1}{\Gamma(n)} \int_0^\infty t^{n-1} M_X(-t) \, dt.
\]
\cite{cressie1981}
\end{proof}

\begin{proof}
    Proof Theorem \ref{t: MGF_accurate}
    We consider \(X\) to be a continuous variable. The three MGF expressions are accurate if \(\mathbb{E}[X^\alpha] - M_X^{(\alpha)}(0) = 0\), in other words, if 
    \[\int_{-\infty}^{\infty} (x^\alpha - D^\alpha e^{tx}) \cdot f_X(x) dx = 0 \iff x^\alpha = D^\alpha e^{tx}\]
    where \(D^\alpha\) the differential operator of order \(\alpha\). Note that we are of course taking derivatives w.r.t. \(t\), not \(x\). The proof will be shown for the Grünwald-left derivative.
    \begin{enumerate}[(i)]
        \item \[\leftindex_{GL}{M}_X^{(\alpha)} = D^\alpha_{GL}(e^{tx})  = \lim_{h \to 0} \frac{1}{h^\alpha} \sum_{k=0}^\infty (-1)^k \binom{\alpha}{k} \exp(x(t - kh))\]
        \[= \exp(xt) \lim_{h \to 0} \frac{1}{h^\alpha} \sum_{k=0}^\infty (-1)^k \binom{\alpha}{k} \exp(-xh)^k\]
        \[= \exp(xt) \lim_{h \to 0} \frac{1}{h^\alpha} (1 - \exp(-xh))^\alpha\]
        (where we used the binomial series identity defined in \ref{d: binomial}). Now we use the Taylor expansion of \(\exp(-xh) = 1 - xh + \frac{(xh)^2}{2!} - \frac{(xh)^3}{3!} + ... - ...\). Since \(h \to 0\), we get: \(\exp(-xh) = 1 - xh + \mathcal{O}(h^2)\) (the rest of the terms are negligible).
        Thus we get:
        \[ \exp(xt) \lim_{h \to 0} \frac{1}{h^\alpha} (1 -(1 - xh + \mathcal{O}(n)))^\alpha = \exp(xt) \lim_{h \to 0} \frac{1}{h^\alpha} (h(x + \mathcal{O}(h)))^\alpha \]
        \[ = \exp(xt) \lim_{h \to 0} \frac{1}{h^\alpha} (h^\alpha((x + \mathcal{O}(h)))^\alpha) = \exp(xt) \lim_{h \to 0} (x + \mathcal{O}(h))^\alpha\]
        \[ = \exp(xt) x^\alpha.\]
        Finally, we take a value of \(t\) around 0 and obtain \(\leftindex_{GL}{M}_X^{(\alpha)}(0) = x^\alpha\) 
        \item We consider the Riemann-Liouville derivative: 
        \[D_{RL}^\alpha(e^{tx}) = \frac{d^n}{dt^n} \frac{1}{\Gamma(n -\alpha)}  \int_{-\infty}^{t} (t-s)^{n - \alpha-1} e^{sx} ds\]
        \cite{koning15} has shown that this derivative is equal to \(x^\alpha e^{tx}\). Thus, if we take a value of \(t\) around 0, we get: \(\leftindex_{RL}{M}_X^{(\alpha)}(0) = x^\alpha\)
    \end{enumerate}

    The proof also holds for the case when \(X\) is a discrete random variable.
\end{proof}

\begin{proof}
    Proof of Theorem \ref{t: MGF_inaccurate}
    We compute \[\leftindex_{CF}{M}_X^{(\alpha)} = D^\alpha_{CF}(e^{tx}) = \frac{d^n}{dt^n}\frac{1}{1 - \beta}  \int_{-\infty}^{t} \exp\left(\frac{-\beta}{1 - \beta}(t - s)\right) x \exp(xs) ds\]
    \[= \frac{d^n}{dt^n}\frac{x}{1 - \beta} \exp\left(\frac{-\beta t}{1 - \beta}\right) \int_{-\infty}^{t} \exp(s\left(\frac{\beta }{1 - \beta} + x \right)) ds  \text{ where } \beta = \alpha - n \text{ and } n = \lfloor \alpha \rfloor.\]
    Now let \(u = \frac{\beta}{1 - \beta} + x\), so we get:
    \[\frac{d^n}{dt^n}\frac{x}{1 - \beta} \exp\left(\frac{-\beta t}{1 - \beta}\right) \int_{-\infty}^{t} \exp(us) ds = \frac{d^n}{dt^n}\frac{x}{1 - \beta} \exp\left(\frac{-\beta t}{1 - \beta}\right) \frac{1}{u} exp(us)\Big|_{-\infty}^{t}\]
    \[= \frac{x}{1 - \beta} \exp\left(\frac{-\beta t}{1 - \beta}\right) \cdot \frac{1}{u} \cdot \exp(u t) = \frac{x \exp(xt)}{(1 - \beta)x + \beta}\]
    Now we apply the derivative of the above expression, with respect to \(t\), \(n\) times:
    \[\frac{d^n}{d t^n} \frac{x \exp(xt)}{(1 - \beta)x + \beta} = \frac{x^{n+1} \exp(xt)}{(1 - \beta)x + \beta}\]
    Now we take a value of \(t\) around 0 and obtain:
    \[\leftindex_{CF}{M}_X^{(\alpha)} = D^\alpha_{CF}(e^{tx}) = \frac{x^{n+1} }{(1 - \beta)x + \beta}.\]
    The latter expression is not equal to \(x^\alpha\), thus the error of \(\leftindex_{CF}{M}_X^{(\alpha)}\) is:
    \[x^\alpha - \frac{x^{n+1} }{(1 - \beta)x + \beta}\]

    
\end{proof}

\newpage

\section{Appendix C }\label{s: app_C}
\renewcommand{\arraystretch}{1.5}
\begin{longtable}{|p{2.5cm}|p{5.5cm}|p{4.5cm}|p{3.0cm}|}
\hline
\textbf{Distribution} & \textbf{PDF / PMF} & \textbf{MGF \(M_X(t)\)} & \textbf{Restrictions} \\
\hline
\endfirsthead
\hline
\textbf{Name} & \textbf{PDF / PMF} & \textbf{MGF \(M_X(t)\)} & \textbf{Restrictions} \\
\hline
\endhead

\( \text{Bernoulli}(p) \) & \( f(x) = p^x(1-p)^{1-x} \) & \( M_X(t) = (1 - p) + pe^t \) & \( 0 \leq p \leq 1, \ x \in \{0,1\}\) \\
\hline

\( \text{Binomial}(n, p) \) & \[ f(x) = \binom{n}{x} p^x (1-p)^{n-x} \] & \( M_X(t) = (1 - p + pe^t)^n \) & \( n \in \mathbb{N},\ 0 \leq p \leq 1 \) \\
\hline

Discrete Uniform  & \( f(x) = \frac{1}{N} \) & \( M_X(t) = \frac{1}{N} \sum_{k=1}^N e^{tk} \) & \( N \in \mathbb{N} \) \\
\hline

\( \text{Geometric}(p) \) & \( f(x) = p(1-p)^{x-1}\) & \( \frac{pe^t}{1 - (1 - p)e^t},\ t < -\ln(1 - p) \) & \( 0 < p \leq 1,\ x \geq 1  \) \\
\hline

\( \text{Poisson}(\lambda) \) & \[ f(x) = \frac{\lambda^x e^{-\lambda}}{x!}\] & \( \exp(\lambda(e^t - 1)) \) & \( \lambda > 0,\ x \geq 0  \) \\
\hline

\( \text{Beta}(\alpha, \beta) \) & \[ f(x) = \frac{1}{B(\alpha, \beta)} x^{\alpha-1}(1-x)^{\beta-1}\] & Messy & \( \alpha, \beta > 0,\ x \in (0,1)  \) \\
\hline

\( \text{Chi-Squared}(p) \) & \[ f(x) = \frac{1}{2^{p/2}\Gamma(p/2)} x^{p/2 - 1} e^{-x/2}\] & \( (1 - 2t)^{-p/2},\ t < 1/2 \) & \( p > 0,\ x > 0  \) \\
\hline

\( \text{Laplace}(\mu, \sigma) \) & \[ f(x) = \frac{1}{2\sigma} e^{-\frac{|x - \mu|}{\sigma}} \] & \( \frac{e^{\mu t}}{1 - (\sigma t)^2},\ |t| < \frac{1}{\sigma} \) & \( \sigma > 0 \) \\
\hline

\( \text{Exponential}(\beta) \) & \( f(x) = \frac{1}{\beta} e^{-x/\beta} \) & \( \frac{1}{1 - \beta t},\ t < 1/\beta \) & \( \beta > 0,\ x > 0 \) \\
\hline

\( \text{Gamma}(\alpha, \beta) \) & \[ f(x) = \frac{1}{\Gamma(\alpha)\beta^\alpha} x^{\alpha-1} e^{-x/\beta} \] & \( (1 - \beta t)^{-\alpha},\ t < 1/\beta \) & \( \alpha, \beta > 0,\ x > 0 \) \\
\hline

\( \text{Logistic}(\mu, \beta) \) & \[ f(x) = \frac{e^{-(x - \mu)/\beta}}{\beta(1 + e^{-(x - \mu)/\beta})^2} \] & \(e^{\mu t}\Gamma(1 - \beta t) \Gamma(1 + \beta t) \)& \( \beta > 0 \) \\
\hline

Normal \( \mathcal{N}(\mu, \sigma^2) \) & \[ f(x) = \frac{1}{\sqrt{2\pi \sigma^2}} e^{-\frac{(x - \mu)^2}{2\sigma^2}} \] & \( \exp\left( \mu t + \frac{\sigma^2 t^2}{2} \right) \) & \( \sigma > 0 \) \\
\hline

\( \text{Uniform}(a, b) \) & \( f(x) = \frac{1}{b - a}\) & \( \frac{e^{tb} - e^{ta}}{t(b - a)} \), \( t \neq 0 \) & \( a < b,\ x \in [a,b]  \) \\
\hline
\end{longtable}\label{t: MGF_Appendix}
\section{Appendix}\label{s:app1}
Place any appendices after the acknowledgments, starting on a new page.
The appendices are numbered
\textbf{A}, \textbf{B}, \textbf{C}, and so forth.
The appendices contain material that you would like to share with the
reader but that would hinder the flow of reading. For instance long
proofs of theorems, code of algorithms, data, etc.

\section{Appendix}\label{s:app2}
You might give more appendices. For instance Appendix \ref{s:app1} for
proofs, Appendix \ref{s:app2} for data.

\section{Appendix}\label{app:quadratic}
The solution to \eqref{eq:quadratic} has the form

\begin{equation} \label{eq:quadraticsol}
x = \frac{-b \pm \sqrt{b^2-4ac}}{2a} \quad\text{if}\;\; a \ne 0.
\end{equation}

%\printbibliography
% or in case of classic bibtex:
\bibliographystyle{chicago}
\bibliography{bib/bibfile}

\end{document}
