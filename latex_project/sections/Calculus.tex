\section{Fractional Calculus}\label{s:calculus}
In order to obtain the expression as mentioned in section \ref{s:intro}, some advanced tools are required. We can find this in the field of fractional calculus.
We define the following:
\begin{definition}
    Let \(D\) be the differential operator, such that \(D f(x) = \frac{d}{dx} f(x)\). Then the fractional derivative of order \(\alpha\) is defined as \(D^{\alpha} f(x) = \frac{d^{\alpha}}{dx^{\alpha}} f(x)\).
\end{definition}
In this definition, \(\alpha\) can be any real number. When taking regular derivatives, \(\alpha \in \mathbb{N}\). In our case, we are interested in instances where  \(\alpha \in \mathbb{Q_{\geq 0}}\).

It is also possible to study derivatives of negative order, which can be used to obtain moments of negative order of a function. A derivative of negative order is simply an integral of positive order. This is defined as follows:
\begin{definition}
    Let \(J\) be the integral operator, such that \(I f(x) = \int f(x) dx\). Then the fractional integral of order \(\alpha\) is defined as \(I^{\alpha} f(x) = \int f(x) dx^{\alpha}\).
\end{definition}

A lot of different definition have been used to compute a fractional derivative. In this paper, we will focus on the following fractional derivatives:
\begin{definition}
    The Riemann-Louville fractional derivative of order \(\alpha\) is defined as:
    \begin{equation}
        D^{\alpha} f(x) =  \frac{d^{n}}{dx^{n}} D_{x}^{-(n - \alpha)} f(x) = \frac{d^{n}}{dx^{n}} I_{x}^{n - \alpha} f(x) = \frac{1}{\Gamma(\alpha)}  \int_{0}^{x} (x-t)^{\alpha-1} f(t) dt
    \end{equation}
    Where \(n = \lceil \alpha \rceil\), the ceiling function and \(\Gamma(.)\) is the Gamma function, see section \ref{s:appendices}.
   
\end{definition}
A modification of the Riemann-Louville derivative is the Caputo derivative, which is defined as follows:
\begin{definition}
    The Caputo fractional derivative of order \(\alpha\) is defined as:
    \begin{equation}
        D^{\alpha} f(x) = \frac{1}{\Gamma(n-\alpha)}  \int_{0}^{x} (x-t)^{n-\alpha-1} f(t) dt
    \end{equation}
    Where, again \(n = \lceil \alpha \rceil\), the ceiling function and \(\Gamma(.)\) is the Gamma function.
    
\end{definition}

Lastly, we will define the Riesz derivative, which is defined as follows:
\begin{definition}
    The Riesz fractional derivative of order \(\alpha\) is defined as:
    \begin{equation}
        D^{\alpha} f(x) = -\int_{-\infty}^{\infty} |\xi|^{\alpha} \hat{f}(\xi) e^{2\pi i  x \xi} d\xi
    \end{equation}
    Where \(\hat{f}(\xi)\) is the Fourier transform of \(f(x)\).
\end{definition}