\section{Introduction}\label{s:intro}
Statistical moments, defined as the \(n\)-th moment of a random variable \(X\) via its probability density function, are essential tools for characterizing data and its distribution. The first and second order moments, the mean and variance, respectively provide key insights into the average value and dispersion of a random variable. Higher-order moments offer further information about the shape and symmetry of the distribution. A less commonly discussed class of moments, however, are the fractional moments. The latter are defined in precisely the same manner as the integer moments, but now with \(n \in \mathbb{R}\), or even \(n \in \mathbb{C}\). From this point onward, when referring to fractional moments, we denote the order by \(\alpha \in \mathbb{R}\) instead of \(n\). While these moments may not find as much usage in comparison with the integer moments of a distribution, they can be rather valuable in specific applications.
\newline
Fractional moments play a significant role in various fields, including finance, economics, and statistics. One notable application in statistics is their use in approximating integer moments, as described by \cite{inverardi2024}. In such cases, evaluating a 'nearby' existing fractional moment can provide an approximation of the otherwise undefined integer moment \cite{inverardi2024}.
\newline
As an illustration, one can consider the student-t distribution with \(\nu = 2\) degrees of freedom. This distribution only has central and raw moments of order \(k\), where \( 0 < k < \nu\). This implies that its second central moment \((k = 2)\) does not exist. One could however consider the central moment of order \(k\), where \(k = 1.95\), given that this order of the moment does in fact exist and interpret its value as the variance of the distribution. 
\newline
Fractional moments are also used in financial modelling, particularly in the context of Generalized Autoregressive Conditional Heteroskedasticity (GARCH) models. These models are commonly applied to time series data such as financial returns and capture the dynamic volatility that changes over time. The GARCH model achieves this by modelling the volatility based on the returns and variances of previous time periods. It is useful to note that in the context of financial returns, the variable of interest often follows a distribution with heavy tails. As a consequence, not all relevant moments of integer order may exist. In such cases, fractional moments may serve as a viable alternative.  \cite{hansen2024} propose a method for computing fractional absolute moments of the cumulative return, quantities that would otherwise be undefined using traditional integer-order techniques. 
\newline 
In addition, \cite{hansen2024} apply a Heterogenous Autoregressive Gamma Model (HARG) to model intraday stock data. The HARG model assumes the conditional distribution of the variable of interest, say \(X_t\) to be of a non-central Gamma distribution. This model is commonly used to model positive-valued time series data \cite{gourierroux2006}. In particular, the volatility of the random variable is of interest, which by definition is non-negative.
\newline
In this context, \(X_t\) is the realized variance. \cite{hansen2024} study the conditional moments of the variance process. Specifically, they consider moments of order \(\{\frac{1}{2}, \frac{3}{2}, 2\}\) which correspond to the standard deviation, skewness and kurtosis respectively. Furthermore, the moment of order \(-\frac{1}{2}\), the inverse volatility has also been computed. In this context, this moment serves as an estimate of the Sharpe ratio, which is an index to measure the performance of some investment compared to a risk-free asset \cite{sharpe1994}. These fractional moments offer valuable insights into financial return dynamics and are highly relevant for modelling and risk assessment. 
\newline
In a related application, \cite{Mikosc2013} apply fractional moments to financial time series data, focusing on the stationary solution of a stochastic recurrence equation - a recursive formula that describes the evolution of a time series. This equation generalizes common time series models in econometrics such as the ARCH and GARCH and is thus of great interest in the context of financial econometrics. Notably, the GARCH(1,1) model can be represented within this one-dimensional stochastic recurrence equation framework \cite{Mikosc2013}. 
\newline
Similar to the case study by \cite{hansen2024}, conditional moments of fractional order are of great interest, as is often the case in the context of (G)ARCH models. The latter can again be explained by the fact that fractional moments offer a valuable alternative for characterizing the distributional properties of financial time series.
\newline
\cite{gyzl2013} also introduce the use of Fractional moments in risk-models, specifically insurance models. In such models, often the probability density function of total ruin, the event where an insurance company's capital becomes negative, is unknown or difficult to compute. The method of maximum entropy has been developed as a way to approximate these unknown density functions. This method uses fractional moments as input, as they have been proven to be able to characterize its distribution \cite{lin1992}. Compared to traditional techniques, such as inverse Laplace transforms, the maximum entropy approach requires fewer moments to achieve a reliable approximation, making it more computationally efficient and practical for real-world applications. 
\newline
In prior research by \cite{DAmico2002}  the same approach of using fractional moments as input for maximum entropy methods to characterize unknown distributions has been employed. One notable application is in option pricing in A Black-Scholes model. Making use of fractional moments, numerical approximations of theoretical entropies have been obtained. Remarkably, using only two fractional moments, the method achieved approximations with three-digit accuracy.
\newline
 This approach outperformed traditional maximum entropy methods that rely on integer moments, both in terms of accuracy and computational efficiency. Moreover, the use of fractional moments helps to avoid numerical optimization issues that commonly arise when minimizing functions based on integer-moment constraints. Therefore, the results obtained by \cite{DAmico2002} seem to correspond with the aforementioned results obtained by \cite{gyzl2013}. Namely, the implementation of fractional moments in the context of maximum entropy methods has great advantages when it comes to numerical stability and efficiency as compared to more traditional methods. 
\newline
Beyond finance and risk modelling, fractional moments have found valuable applications in fields such as engineering and healthcare. As previously mentioned, \cite{gyzl2024} used fractional moments combined with the maximum entropy method to predict total ruin. This approach has been extended to estimating lifetime distributions, a general topic of interest in survival analysis \cite{clark2003}. Both probability density functions and survival functions were estimated using this technique. Once again, the prediction errors remained within three-digit accuracy, further supporting the method’s consistency and precision across different fields of research.
\newline
Other examples in the field of engineering include optimizing signal processing and control systems as well as studying the response characteristics of random vibration systems. \cite{wang2025} have shown that when using the concept of fractional moments for the latter, accuracy and stability is higher compared to traditional methods, such as Taylor expansions. Furthermore, in terms of simplicity and efficiency, the method of fractional moments is advantageous, as its computation steps are straightforward and avoid convergence issues, significantly reducing the resources required for computation. Implementing the usage of fractional moments has allowed \cite{wang2025} to obtain both analytical, as well as numerical solutions to problems within their research field, which again proves its viability. 
\newline
Another noteworthy application of fractional moments is in the identification of distributions in complex, non-linear systems. \cite{dimatteo2014} demonstrate how complex fractional moments can be used to solve differential equations such as the Kolmogorov or Fokker-Planck equation, which arise in the context of continuous-time Markov processes.
By transforming the system into a more agreeable form using Mellin transforms, \cite{dimatteo2014} show that solutions can be efficiently computed and later recovered using inverse transformations. A key advantage of this method is that it preserves the entire support of the probability distribution, which traditional integer moment methods fail to achieve.
\newline
The focus of this paper is on combining the theory of fractional calculus and moment generating functions in order to obtain moments of fractional order. Comparisons between the accuracies of the moment generating function in combination with different fractional derivatives will be made. The interpretability and advantages of moments of fractional order compared to moments of integer order will also be discussed. 
The remainder of this paper is structured as follows.
Section \ref{ss:methodology_introduction} discusses existing methods for computing fractional moments, outlining their respective advantages and disadvantages. This section concludes with an introduction to the novel approach proposed in this paper, the required techniques, and the limitations to be mindful of. Section \ref{s:calculus} lays the mathematical foundation for this new method and provides a brief historical overview of the development of fractional derivatives. In section \ref{s:MGF}, some key definitions and properties of statistical moments and the moment generating function will be highlighted. What is more, the theory of fractional derivatives from section \ref{s:calculus} will be incorporated in order to extend functionality of the moment generating function. Relevant properties are revisited in this new context, and potential analytical errors of the method are addressed. In section \ref{s:simulation},  simulation results will be analysed, which further clarify the numerical stability and potential errors of computing fractional moments via the moment generating function. Finally, in section \ref{s:practical_case}, a practical implementation of this method in a financial setting will be considered. The results will be compared against those obtained using traditional methods, with a focus on the implications of potential errors in real-world applications.