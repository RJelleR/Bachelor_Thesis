\section{Introduction}\label{s:intro}
Statistical moments, defined as the \(n\)-th moment of a random variable \(X\) via its probability density function, are essential tools for characterizing data and its distribution. The first and second (central) order moments, the mean and variance, respectively provide key insights into the average value and dispersion of a random variable. Higher-order moments offer further information about the shape and symmetry of the distribution. A less commonly discussed class of moments, however, are the fractional moments. The latter are defined in precisely the same manner as the integer moments, but now with \(n \in \mathbb{R}\), or even \(n \in \mathbb{C}\). From this point onward, when referring to fractional moments, we denote the order by \(\alpha \in \mathbb{R}\) instead of \(n\). While these moments may not find as much usage in comparison with the integer moments of a distribution, they can be rather valuable in specific applications.

Fractional moments play a significant role in various fields, including finance, economics, and statistics. One notable application in statistics is their use in approximating integer moments, as described by \citet{inverardi2024}. In such cases, evaluating a 'nearby' existing fractional moment can provide an approximation of the otherwise undefined integer moment \citep{inverardi2024}. As an illustration, one can consider the student-t distribution with \(\nu = 2\) degrees of freedom. This distribution only has central and raw moments of order \(k\), where \( 0 < k < \nu\). This implies that its second central moment \((k = 2)\) does not exist. One could however consider the central moment of order \(k\), where \(k = 1.95\), given that this order of the moment does in fact exist and interpret its value as the variance of the distribution. 

A different theoretical application is provided by \citet{Mikosc2013}, who analyse fractional moments in the context of stationary solutions to stochastic recurrence equations (SREs). These equations generalize standard econometrics models like Autoregressive Conditional Heteroskedasticity (ARCH) and Generalized Autoregressive Conditional Heteroskedasticity (GARCH), which are used to monitor volatility of time series data. In particular, the GARCH(1,1) model can be expressed as an SRE, making the analysis of fractional moments in this context relevant for understanding the long-run behavior of volatility models. \citet{Mikosc2013} analyse the existence and behavior of fractional moments from a recursive perspective and characterizing the distributional properties of financial time series. 
\newline
In such theoretical studies, it is often important to determine whether a fractional moment exists, even if its exact value is not required. However, in more applied settings, particularly in finance, and engineering, it becomes necessary to explicitly compute fractional moments. This is the main topic we focus on in this thesis: we will not only explore the theoretical relevance of fractional moments, but also analyse the practical importance of computing them.

One important example arises again in the context of financial modeling using GARCH models. These models are commonly applied to financial return series and capture the dynamic volatility that changes over time. The GARCH model achieves this by modelling the volatility based on the returns and variances of previous time periods. It is useful to note that in the context of financial returns, the variable of interest often follows a distribution with heavy tails. As a consequence, not all relevant moments of integer order may exist. In such cases, fractional moments may serve as a viable alternative.  \citet{hansen2024} propose a method for computing fractional absolute moments of the cumulative return, quantities that would otherwise be undefined using traditional integer-order techniques. 

In addition, \citet{hansen2024} apply a Heterogenous Autoregressive Gamma Model (HARG) to model intraday stock data. The HARG model assumes the conditional distribution of the variable of interest, say \(X_t\) to be of a non-central Gamma distribution. This model is commonly used to model positive-valued time series data \citep{gourierroux2006}. In particular, the volatility of the random variable is of interest, which by definition is non-negative.

In this context, \(X_t\) is the realized variance. \citet{hansen2024} study the conditional moments of the variance process. Specifically, they consider moments of order \(\{\frac{1}{2}, \frac{3}{2}, 2\}\) which correspond to the standard deviation, skewness and kurtosis respectively. Furthermore, the moment of order \(-\frac{1}{2}\), the inverse volatility has also been computed. In this context, this moment can be used as the inverse of the covariance matrix in order to obtain the Sharpe ratio, which is an index to measure the performance of some investment compared to a risk-free asset \citep{sharpe1994}. These fractional moments offer valuable insights into financial return dynamics and are highly relevant for modelling and risk assessment. 

Furthermore, computing fractional moments becomes essential in entropy-based density reconstruction. \citet{gyzl2013} introduce the use of fractional moments as input for maximum entropy methods in the context of insurance risk modelling. In these models, often the probability density function of total ruin, the event where an insurance company's capital becomes negative, is unknown or difficult to compute. The method of maximum entropy has been developed as a way to approximate these unknown density functions by selecting the "least biased" probability density based on a given set of moments. Traditionally, this method relies on moments of integer order. However, \citet{gyzl2013} demonstrate that fractional moments can be used instead to obtain the unknown ruin distribution function more efficiently and accurately, as is supported by \citet{lin1992}, for real-world applications. Moreover, as suggested before, when it is impossible to take high-order integers moments, which may be required to approximate the unknown density function, moments of fractional order serve as a valuable alternative. 

\citet{DAmico2002} apply a similar approach in the context of option pricing using the Black-Scholes model, a differential equation which can be used to determine the price of an option. They use fractional moments to approximate the underlying density of th asset price and compute related entropic measures. Most importantly, they show that the use of only two fractional moments is sufficient to approximate the entropy and associated distribution at a high accuracy, outperforming the classical integer-moment-based maximum entropy methods both in numerical stability and computational efficiency. Remarkably, using only two fractional moments, the method achieved approximations with three-digit accuracy.

 These applications demonstrate that the computation of fractional moments plays a practical and important role in reconstructing risk-relevant distributions via entropy-based methods. This approach is particularly useful in finance and insurance contexts, where the underlying distribution is often unknown, and entropy-based reconstruction using fractional moments provides a reliable and efficient alternative to traditional techniques such as inverse Laplace transforms (see definition \ref{def:laplace}).
\newline
Beyond finance and risk modelling, computation of fractional moments finds practical applications in fields such as engineering and healthcare. As previously mentioned, \citet{gyzl2013} used fractional moments combined with the maximum entropy method to predict total ruin. This approach has been extended to the context of estimating lifetime distributions \citep{gyzl2024}, a general topic of interest in survival analysis \citep{clark2003}. Both probability density functions and survival functions were estimated using this technique. As before, the prediction errors remained within three-digit accuracy, further supporting the method’s consistency and precision across different fields of research. The same practical advantages of computing moments of fractional order over moments of integer as in \citet{gyzl2013} still apply.

Other examples include optimizing signal processing and control systems as well as studying the response characteristics of random vibration systems. \citet{wang2025} have shown that using the concept of fractional moments for the latter leads to higher  accuracy and greater numerical stability compared to traditional methods, such as Taylor expansions. Furthermore, in terms of simplicity and efficiency, the method of fractional moments is advantageous, as its computation steps are straightforward and avoid convergence issues, significantly reducing the resources required for computation. Implementing the usage of fractional moments has allowed \citet{wang2025} to obtain both analytical, as well as numerical solutions to problems within their research field, which again proves its viability in practical applications. 

Finally, in the context of non-linear systems, \cite{dimatteo2014} demonstrate how complex fractional moments can be used to solve differential equations such as the Kolmogorov or Fokker-Planck equation, which are often used in the field of statistical mechanics to model processes such as diffusion, relaxation to equilibrium, and entropy production \citep{schulten2005}. By transforming the system into a more agreeable form using Mellin transforms (see definition\ref{def:mellin}), \citet{dimatteo2014} show that solutions can be efficiently computed and later recovered using inverse transformations. A key advantage of this method is that it preserves the entire support of the probability distribution, which traditional integer moment methods fail to achieve.
\newline
In this thesis, we propose a novel method to compute moments of fractional order by combining the moment generating function with techniques from fractional calculus. Both analytical and numerical comparisons between the accuracies of the moment generating function in combination with different fractional derivatives will be made.  

The remainder of this thesis is structured as follows.
Section \ref{ss:methodology_overview} reviews existing methods for computing fractional moments and introduces a novel approach based on combining the moment generating function with fractional calculus. Section \ref{s:calculus} and section \ref{s:MGF} lay the theoretical foundation, including the definitions and properties of fractional derivatives and moment generating functions. Section \ref{s:simulation} presents numerical results which will be used to compare the accuracy and potential errors of different methods. Finally, in section \ref{s:practical_case}, a practical implementation of this new method in a financial setting will be considered, with a focus on the implications of potential associated errors in real-world applications.