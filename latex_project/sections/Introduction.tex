\section{Introduction}\label{s:intro}
Statistical moments, defined as the \(n\)-th moment of a random variable \(X\), with probability density function (PDF) \(f_X(x)\) 
\[
\mathbb{E}[X^n] = 
\begin{cases} 
\int_{-\infty}^{\infty} (x - c)^n f_X(x) \, dx & \text{if } f_X(x) \text{ is continuous,} \\ 
\sum_{i} (x_i - c_ i)^n f_X(x_i) & \text{if } f_X(x) \text{ is discrete.} 
\end{cases}, n \in \mathbb{N}
\] are essential tools to characterize data and its distribution. Moments of the first and second order, the mean and variance respectively, provide one with essential information about the average and measure of dispersion of a random variable. Moments of even higher order are useful regarding the shape and symmetry of the distribution. Less known moments, however, are the fractional moments. The latter are defined in precisely the same manner as the integer moments, but now with \(n \in \mathbb{R}\), or even \(n \in \mathbb{C}\). From this point on, when fractional moments are considered, we denote \(\alpha \mathbb{C}\), instead of \(n\), to be its moment's order. While these moments may not find as much usage in comparison with the integer moments of a distribution, they can be very useful in certain applications.
\newline
Fractional moments play a significant role in a variety fields, including finance, economics, and statistics An example of the latter is its application in approximating integer moments as described by \cite{inverardi2024}. This is especially useful when (for \(\alpha \in \mathbb{R}\)) the \(\lceil\alpha \rceil\)-th central integer moment might not exist, while its fractional moment does. Finding an existing fractional moment close to a non-existing integer moment, could still provide one with information about this integer moment. For example, the student-t distribution with \(\nu = 2\) degrees of freedom only has central and raw moments of order \(k\), where \( 0 < k < \nu\). This implies that its second central moment \((k = 2)\) does not exist. One could however consider the \(k-th\) central moment where \(k = 1.95\) and interpret its value as the variance of the distribution. Fractional moments are also used in financial modelling, particularly in the context of Generalized Autoregressive Conditional Heteroskedasticity (GARCH) models. These models are commonly applied to time series data such as financial returns and capture the dynamic volatility that changes over time. The GARCH model achieves this by modeling the volatility based on the returns and variances of previous time periods. \cite{hansen2024} have obtained a method of finding fractional absolute moments of the cumulative return, which would have been impossible when using any other method. \cite{gyzl2013} have also introduced the usage of Fractional moments in risk-models, specifically insurance models. In such models, often the probability density function of total ruin, the event where an insurance company's capital becomes negative, is unknown. The so-called "Method of maximum entropy" has been developed to find these densities. This method takes fractional moments as its input, as they have been proven to be able to characterize its distribution \cite{lin1992}. This method has proven to be a useful alternative to existing methods, such as inverse Laplace transformations. This is the case as this new method takes less values than the latter as input, making it computationally more efficient. Beyond finance and risk modelling, fractional moments also have important applications in engineering. Examples include optimizing signal processing and control systems as well as studying the response characteristics of random vibration systems. \cite{wang2025} has shown that when using the concept of fractional moments for the latter, accuracy and stability is higher compared to traditional methods, such as Taylor expansions. Furthermore, in terms of simplicity and efficiency, the method of fractional moments is advantageous, as its computation steps are straightforward and avoid convergence issues, significantly reducing the resources required for computation. Working with fractional moments has allowed \cite{wang2025} to obtain both analytical, as well as numerical solutions to problems within their research field, which again proves its viability. Another application within the field of engineering, can be found in the identification of distributions of non-linear systems. \cite{dimatteo2014} have shown that complex fractional moments allow one to solve equations such as the Kolmogorov or Fokker-Planck equation, a characterization of continuous-time Markov processes. After performing a Mellin transformation on this system of equations, the resulting system is a linear system in terms of complex fractional moments. The latter can now be solved rather easily and taking the Inverse Mellin transformation on these solutions immediately provides one with the solutions of the non-linear system. Advantages of using complex fractional moments instead of integer moments is that, when applying the Mellin transformation, the relevant PDF is restored on its entire support. This is not the case for the integer moments. This method of using complex fractional moments has been proven to have a rather high accuracy and is applicable to any stable kind of non-linear system of equations \cite{dimatteo2014}.