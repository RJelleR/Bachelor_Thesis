\section{Methodology}\label{s:methodology}
In this section we present the methodological framework for computing fractional moments using different approaches. We begin by reviewing existing methods before introducing an alternative approach based on fractional calculus. The comparative structure of the analysis, both theoretical and empirical, is also outlined. 

\subsection{Overview of existing methods and introduction to a novel approach}

In this subsection we review the traditional method of computing fractional moments and a modern alternative based on the complex moment generating function (CMGF). What is more, we also introduce the motivation for using fractional derivatives applied to the moment generating function, which forms the foundation of this thesis’ novel approach.

\subsubsection{Existing methods of obtaining fractional moments}\label{ss:methodology_introduction}
The formal definition of the statistical moments mentioned in the previous section is as follows: The \(n\)-th moment of a random variable \(X\), with probability density function (PDF) \(f_X(x)\) 
\[
\mathbb{E}[X^n] = 
\begin{cases} 
\int_{-\infty}^{\infty} (x - c)^n f_X(x) \, dx & \text{if } f_X(x) \text{ is continuous,} \\ 
\sum_{i} (x_i - c_ i)^n f_X(x_i) & \text{if } f_X(x) \text{ is discrete.} 
\end{cases}, n \in \mathbb{N}
\] \cite{feller1957}

The traditional method of computing fractional moments is rather straightforward. Similar to integer moments, one simply computes the summation (in the discrete case) or integral (in the continuous case) of \(x^\alpha \cdot f_x(x)\), where \(f_x(x)\) denotes the probability density function of the random variable \(X\). In the context of fractional moments, \(\alpha \in \mathbb{R}\) instead of \(\mathbb{Z}\) (assuming that negative moments exist). \cite{hansen2024} introduce an alternative approach to computing fractional moments using the complex moment generating function (CMGF), which they apply in the context of the aforementioned GARCH models. One of their expressions of importance is given by:

\[\mathbb{E}\left| X - \xi \right|^\alpha = \frac{\Gamma(\alpha+1)}{2\pi} \int_{-\infty}^{\infty} \frac{e^{-\xi z} M_X(z) + e^{\xi z} M_X(-z)}{z^{\alpha+1}} dt\] \(\text{ where } z = s + it, s \in \mathbb{N_+}, \xi \in \mathbb{R} \) and \(\alpha\) of course the order of the moment.

This formulation extends upon the traditional moment generating function (MGF) but avoids the process of taking derivatives, making it computationally efficient. The inclusion of the Gamma function is logical, as it extends the factorial function to real values, aligning well with the computation of fractional moments. Since this method relies on integration, rather than differentiation, it avoids numerical issues that might arise when computing derivatives, such as obtaining rather great approximation errors.

\subsubsection{Obtaining fractional moments by using the moment generating function}
While the CMGF method provides an efficient  and elegant alternative to the traditional method, this thesis explores a different approach: computing fractional moments directly by applying fractional derivatives to the MGF. The MGF is widely used for computing integer moments by differentiation around zero. Extending this approach to fractional orders requires us to take fractional derivatives. Thus, we need to define such fractional derivative operators. These fractional derivatives have a long history and often make use of the aforementioned Gamma function in combination with some integral. This means that, for continuous random variables, where we integrate the MGF, we will have to do double integration. A consequence might be that obtaining analytical expressions of these moments may not always be possible. A great number of alternative expressions of these fractional derivatives have been created, mostly based on different interpretations of the latter in the field of physics. In this thesis, we will focus on computing the MGF using the Riemann-Liouville derivative, the Caputo-Fabrizio derivative and the Grünwald-Letnikov fractional derivative. Each of these fractional derivatives in combination with the MGF might lead to different moments expressions for the same distribution and same fractional order of the moment. Thus, it is essential to compare each of these definitions with the traditional way of computing fractional moments, to derive their accuracy and conclude which approach is most suitable for fractional moment computation. Similar to the expression of the moments of a random variable, their errors may also be hard to derive analytically, depending on its distribution.

\subsubsection{Order and methods of research}
In order to compare the aforementioned fractional derivatives, the order of research will be as follows. First, a mathematical groundwork for the fractional derivatives will be laid, in which a number of their respective properties will be discussed. Next, we will revise some basic definitions of statistical moments and the moment generating function and see how some of these properties change when considering fractional moments in combination with the moment generating function. If possible, analytical expressions of the errors for each fractional derivative in combination with the MGF will be computed, which will conclude the theoretical research of this thesis. We will conduct a simulation using the programming language Julia, which is well-known for its focus on data science and fast computations, to now obtain numerical errors instead of analytical errors. The simulation will evaluate each MGF-based method using statistical measures, including minimum, maximum, and average errors. Finally, we will consider a practical case; analyzing volatility and risk of the S\&P 500 returns using fractional moments in which all three derivatives will once again be compared.

\section{Fractional Calculus}\label{s:calculus}
\subsection{Overview and applications of fractional derivatives}\label{ss:calculus_introduction}
Although not the main topic of interest of this thesis, it is useful to have some knowledge about the history and applications of fractional derivatives. The study of fractional derivatives has been relevant as early as the year 1695 when the concept of such a derivative was implicitly discussed by Leibniz and Bernoulli \cite{katugampola2014}. Since then, numerous definitions of fractional derivatives have been developed. The best-known definition of the fractional derivative is the Riemann-Liouville derivative, its upper derivative of a function \(f(x)\) of order \(\alpha\) is denoted as 
\[\frac{d^n}{dx^n}\frac{1}{\Gamma(n -\alpha)} \int_{a}^{x} (x-t)^{n - \alpha -1} f(t) dt, \text{ where }n = \lceil\alpha \rceil \] \cite{kilbas2006}.
Michele \cite{caputo1967} defined a variation on this derivative, where instead of \(\frac{d^n}{dx^n}\) in front of the integral, we have \(\frac{d^n}{dt^n}\) inside the integral. Due to this adjustment, it is possible to have initial value conditions expressed as the traditional derivatives of integer-order, which made these fractional differential equation problems more intuitive. Other fractional derivatives, such as the \cite{hadamard1892} and \cite{riesz1949} derivative, have been defined to take advantage of particular beneficial properties. For example, each of the latter derivatives can be written as a Fourier transformation. As a consequence, the analytical expressions, can often be simplified. A rather unique derivative is the Grünwald-Letnikov derivative which, in contrast to all the aforementioned derivatives, is not based on integral. Instead, it generalizes the difference quotient, \(\frac{f(x+h) - f(x)}{h}\),  to fractional orders using binomial coefficients \cite{atici2021}. This variety of definitions emphasises how dependent fractional derivatives are  on different physical interpretations and practical applications.
Beyond their theoretical significance, fractional derivatives have been of significant importance in various scientific fields since the 19th century. Examples include fractional Fourier transformations, a generalization of the regular Fourier transformations \cite{missbauer2012}, fractional diffusion equation models, describing the motion of particles in liquids as a consequence of thermal molecular motions \cite{einstein1905} and the fractional Schrödinger equation, a generalization of the Schrödinger equation, often used in quantum mechanics \cite{laskin2002}. Their applications are less common in the fields of finance or economics, as fractional derivatives are mainly used to describe natural phenomena \cite{boulaaras2023}. Yet they still offer some great potential. (Symmetric) Levy flights make use of fractional derivatives in order to solve partial differential equations which describe random walk processes in time series \cite{scalas2000}. The development of fractional derivatives also led to the notion of fractional Brownian motions, a generalization of the Brownian motion \cite{mandelbrot1968}. The latter is a continuous-time stochastic process which, similar to Levy flights, may be used to model random walk processes.
\subsection{Formal definitions of fractional derivatives}
In order to obtain the expressionS mentioned in \autoref{s:methodology}, some advanced tools are required. We can find these in the field of fractional calculus.
We define the following:
\begin{definition}
    Let \(D\) be the differential operator, such that \(D f(x) = \frac{d}{dx} f(x)\). Then the fractional derivative of order \(\alpha\) is defined as \[D^{\alpha} f(x) = \frac{d^{\alpha}}{dx^{\alpha}} f(x)\].
\end{definition}
In this definition, \(\alpha\) can be any real number. When taking regular derivatives, \(\alpha \in \mathbb{N}\). In most of our cases, we are interested in the instance where  \(\alpha \in \mathbb{R}_+\).

It is also possible to study derivatives of negative order, which can be used to obtain moments of negative order of a function, provided that such an order exists. A derivative of negative order is simply an integral of positive order. This is defined as follows:
\begin{definition}
    Let \(I\) be the integral operator, such that \(I f(x) = \int f(x) dx\). Then the fractional integral of order \(\alpha\) is defined as \[(I^{\alpha} f) (x) = \frac{1}{(\alpha-1)!}\int (x-t)^{\alpha-1} f(t) dt\] \cite{cauchy1823}.
\end{definition} 

Combining the previous two definitions, we obtain the following, more general, definition.
\begin{definition}\label{d: differintegral}
    The differintegral operator is defined as
    \begin{equation}
        R^\alpha f(x) = \begin{cases}
            I^{|\alpha|} f(x) & \text{if } \alpha < 0 \\
            D^\alpha f(x) & \text{if } \alpha > 0 \\
            f(x) & \text{if } \alpha = 0
        \end{cases}, \text{ with } \alpha \in \mathbb{R}.
        \end{equation}
\end{definition}

As mentioned in section \ref{ss:calculus_introduction}, quite a number of different definitions have been proposed to compute a fractional derivative. Some of these definitions are rather similar, thus, in this paper, we will focus on some of the more well known fractional derivatives. We will start with the most famous fractional derivative, which laid the foundation of the study of fractional derivatives as early as in 1832.
\begin{definition}
    The left-side Riemann-Liouville fractional derivative of order \(\alpha\) is defined as:
    \begin{equation}
        D^{\alpha}_{a_+} f(x) =  \frac{d^{n}}{dx^{n}} D_{x}^{-(n - \alpha)} f(x) = \frac{d^{n}}{dx^{n}} I_{x}^{n - \alpha} f(x) = \frac{d^n}{dx^n} \frac{1}{\Gamma(n -\alpha)}  \int_{a}^{x} (x-t)^{n - \alpha-1} f(t) dt
    \end{equation} \cite{liouville1832}.

    Here, \(n = \lceil \alpha \rceil\), the ceiling function and \(\Gamma(.)\) is the Gamma function, see section \autoref{s:appendices}.

    Note that we just defined the left-side Riemann-Liouville fractional derivative, suggesting that there also exists a right side derivative. In the case of the latter, we would evaluate the associated integral the other way around. Namely, \(D^{\alpha}_{b_-} f(x) = \frac{d^n}{dx^n} \frac{1}{\Gamma(n -\alpha)}  \int_{x}^{b} (x-t)^{n - \alpha-1} f(t) dt\). We intend on using the left-side derivative, as is supported by \cite{tarasov2023}. The reason is, due to the fact that many functions in probability theory, most importantly the cumulative distribution function, are defined as an integral from some constant to \(x\). 
    
    \begin{remark}\label{r: integer}
        For values \(\alpha \in \mathbb{N}_+, n =  \lceil \alpha \rceil = \alpha\), so \(\Gamma(n - \alpha) = \Gamma(0)\), which is undefined. Thus for \(\alpha \in \mathbb{N}_+\), we define: \(D^\alpha f(x) = \frac{d^\alpha}{dx^\alpha} f(x)\), which is simply the regular expression for derivatives of integer order.
    \end{remark}
   
\end{definition}
A modification of the Riemann-Liouville derivative is the Caputo-Fabrizio derivative, which is defined as follows:
\begin{definition}\label{d: CF}
    The Caputo-Fabrizio fractional derivative of order \(\alpha, \alpha \in [0,1)\) is defined as:
    \begin{equation}
        D^{\alpha} f(x) = \frac{1}{1 - \alpha}  \int_{a}^{x} \exp\left(\frac{-\alpha}{1 - \alpha}(x-t)\right) f'(t) dt
    \end{equation} \cite{caputo2015}.
    With \(a \in [-\infty, x)\).
    The Caputo-Fabrizio is always defined as the integral from some constant to the variable \(x\). This is another reason for choosing to work with the left-side Riemann-Liouville integral. In this way, it will be more straightforward to compare the two integrals. What is more, note that for the Caputo-Fabrizio derivative, the order \(\alpha \in [0, 1)\). This does not mean, however, that one can only compute fractional derivatives of order 1 or lower. There exists a rather convenient property of the differintegral operator which allows one to combine orders of derivatives, which will be discussed in a moment.
    
\end{definition}

Lastly, we will define the Grünwald-Letnikov derivative, which is defined as follows:
\begin{definition}
    The Grünwald-Letnikov fractional derivative of order \(\alpha\) is defined as:
    \begin{equation}
        D^\alpha f(x) = \lim_{h \to 0} \frac{1}{h^\alpha} \sum_{k=0}^\infty (-1)^k \binom{\alpha}{k} f(x - k h)
    \end{equation}
   Where \(\binom{\alpha}{k}\) is the binomial coefficient, with \(0 \leq k \leq \alpha\), see section \autoref{s:appendices}.
\end{definition} \cite{zhmakin2022}.
It is immediately clear, observing the summmation symbol instead of the integral, that this derivative behaves quite differently from the two derivatives defined above. The Grünwald-Letnikov derivative is an extension on derivatives based of the concept of finite differences \cite{flajolet1995}.

We will consider a number of properties which come in useful when working with fractional derivatives.

\begin{proposition}\label{p: calculus}
    The fractional derivatives above adhere to the following properties:
    \begin{enumerate}[(i)]
        \item Linearity: Let \(f(x), g(x)\) be functions and \(a, b, x \in \mathbb{R}\). Then we have that \(D^{\alpha} (a f(x) + b g(x)) = a D^{\alpha} f(x) + b D^{\alpha} g(x)\).
        \item \(D^{\alpha} f(x) = f(x)\), for \(\alpha = 0\) 
        \item for sufficiently smooth functions f, we have that \(D^{\alpha + \beta} f(x) = D^\alpha(D^\beta f(x)) =  D^\beta(D^\alpha f(x))\), with \(\alpha, \beta \in \mathbb{R}\). Note that, for definition \ref{d: CF}, this property only holds for \(\beta \in \mathbb{N}, \alpha \in [0,1)\).
    \end{enumerate}
        
    
\end{proposition}

The proofs of these of the properties stated in this proposition can be found in appendix \ref{s:app_B}. Most of these proofs have been provided by myself, while some other proofs, which are outside the scope of thesis are based on other papers. Note that the third property is especially useful for the Caputo-Fabrizio derivative. This property allows one to take fractional derivatives of order greater than 1 by comparing fractional derivatives and regular integer derivatives.

We will now provide two numerical examples of these fractional derivatives. For simplicity, we will let \(a = 0\):
\begin{example}
    \begin{enumerate}[(i)]
        \item 
    
    We consider the Riemann-Liouville derivative of order \(\frac{3}{2}\) for some constant \(c \in \mathbb{R}\):
    \[D^{\frac{3}{2}}_{RL}(c) = \frac{d^2}{dx^2} \frac{1}{\Gamma(2 - \frac{3}{2})}  \int_{0}^{x} (x-t)^{2 - \frac{3}{2}-1} c dt\]
    \[= \frac{d^2}{dx^2} \frac{c}{\sqrt{\pi}}  \int_{0}^{x} (x-t)^{- \frac{1}{2}} dt = \frac{d^2}{dx^2} \frac{-2c}{\sqrt{\pi}} \sqrt{x - t} \Big|_{0}^{x}\]
    \[= \frac{d^2}{dx^2} \frac{2c \sqrt{x}}{\sqrt{\pi}} = \frac{-c}{2x\sqrt{\pi x}} \neq 0.\] As stated, the fractional derivative of a constant is not equal to zero when using the Riemann-Liouville derivative. This is also the case for the Grünwald-Letnikov derivative, but not for the Caputo-Fabrizio derivative.
    \item We compute the semi-derivative of \(\frac{x}{2}\) using the Caputo-Fabrizio derivative:
    \[D^{\frac{1}{2}}_{CF}\left(\frac{x}{2}\right) = \frac{1}{1 - \frac{1}{2}}  \int_{0}^{x} \exp\left(\frac{-\frac{1}{2}}{1 - \frac{1}{2}}(x-t)\right) \frac{1}{2} dt = \int_{0}^{x} \exp(t - x) dt\]
    \[ =  \exp(t - x) \Big|_{0}^{x} = 1 - \exp(- x).\]
    For \(x \geq 0\), this expression is equal to the CDF of the exponential distribution with \(\lambda = 1\). A remarkble result.
    
    \end{enumerate}
\end{example}
The observant reader might notice that no explicit examples of the Grünwald-Letnikov derivative have been provided. The latter is due to the fact that is rather difficult to obtain analytical expressions for this derivative. Thus, later on in this thesis, when computing fractional moments and their assiocated biases, the main focus for the Grünwald-Letnikov derivative will be on its numerical computations.
\begin{remark}
    
As shortly mentioned in the introduction, it is possible to generalize the order of derivatives even further, extending \(\alpha\) to be in \(\mathbb{C}\) instead of \(\alpha \in \mathbb{R}\). This means that, when combining such derivatives with the moment generating function, we will obtain complex moments. Since statisical moments of complex order do almost not find any usage in applications, as they lack interpretability, they are not the main focus of this research. For the interested reader, \cite{love1971} has done some impressive research on the fundamentals of derivatives of complex order. The obtained expressions are somewhat similar to those of the fractional derivatives which have been discussed above.
\end{remark}

\subsection{The Moment Generating Function of fractional order}\label{s:MGF}
Having defined the techniques required to compute fractional derivatives, we now try to combine them with the MGF. Before proceeding, we will briefly review the formal definition of a moment in statistics.
\subsubsection{Moments}
Recall the definition of statistical moments as in definition \ref{eq:moments} of section \ref{ss:methodology_introduction}. If \(c = \mu_x\), where \(\mu_x\) denotes the expected value of \(X\), then our higher moments are called central moments. In the context of this research, we will focus on the case \(c = 0\), corresponding to raw moments of a random variable \(X\). This choice has been made as the MGF, which we will soon define, only computes raw moments of higher order. A moment of order \(\alpha\) is said to exist, if \(\mathbb{E}[X^\alpha] < \infty\).

\subsubsection{The Moment Generating Function}
We now formally introduce the moment Generating Function, one of the most significant subjects of this thesis.
\begin{definition}
    The moment generating function of a variable \(X\), is defined as
    \[M_X(t) = \mathbb{E}[e^{tX}]\] provided that \(\mathbb{E}[e^{tX}] < \infty\), for all  \( t \in (- h, h)\), which contains 0, for some \(h > 0\) 
\end{definition}

\begin{remark}
    Deriving the expression \(M_X(t) = \mathbb{E}[e^{tX}]\) is typically straightforward. Generally, it simply requires a handful of steps of analytic evaluation. This procedure is not that interesting nor relevant to this research. Thus, when making use of  expressions of the MGF, we will simply refer to the distribution table in Appendix \ref{s:app_common_distributions}.
\end{remark}

 
We will state the theorem which makes the MGF so useful. This theorem allows us to compute moments of higher order by taking derivatives of the given order instead of integrals.
\begin{theorem}\label{t:mgf}
    If \(M_X(t)\) exists on some interval \((-h, h)\), as defined before, we have that:
    \[ \mathbb{E}[X^n] = M_X^{(n)}(0), \text{ for } n \in \mathbb{N}\] 
\end{theorem} 




We introduce the following well-known properties for the MGF \(M_X(t)\):
\begin{proposition}\label{p: moments}
    For \(X, Y\) random variables, we have that:
    \begin{enumerate}[(i)]
        \item \(M_X^{(0)}(t) = \mathbb{E}[e^{0X}] = 1\). This property can be used to confirm that a given function is a valid probability density function (i.e., integrates to one).
        \item Location scale-transform. Assuming \(M_X(t)\) exists, for constants \(\mu, \sigma \in \mathbb{R}\), we have that: 
        \[M_{\mu + \sigma X}(t) = e^{\mu t} \cdot M_X(\sigma t)\]
        \item If \(X \perp Y\), then \(M_{X+Y}(t) = M_X(t)\cdot M_Y(t)\).
    \end{enumerate}
\end{proposition}


The proofs of the latter can be found in Appendix \ref{pf:MGF}.

There are several additional topics closely related to the MGF, including Fourier transforms, Laplace transforms, Wick rotations, and characteristic functions. While these subjects are relevant to the theoretical foundation of the MGF , they fall outside the scope of this thesis and will therefore not be discussed. Readers interested in exploring these concepts further may find \citet{kolmogorov1999} to be a valuable resource.

\subsubsection{Computing moments of negative order using the Moment Generating Function}
In specific cases, as were mentioned in section \ref{s:intro}, moments of negative order can be of interest to characterize the data. Such an example was the moment of order \(-\frac{1}{2}\), which in some contexts may be used to obtain the Sharpe Ratio. Thus, it is useful to understand how to compute moments of such orders. We will consider the continuous case:

\[\mathbb{E}[X^{-n}] = \int_{-\infty}^{\infty} x^{-n} f_X(x) dx = \int_{-\infty}^{\infty} \left(\frac{1}{x}\right)^n f_X(x) dx.\] We immediately observe a rather obvious problem. This integral is undefined at \(x = 0\) and diverges in a neighbourhood around zero. \citet{khuri2002} have stated the following corollary for the existence of a moment with negative first order:
\begin{corollary}
    If \(f_X(x)\) is a continuous pdf defined on \((-\infty, \infty)\), and if \[\lim_{x \to 0} \frac{f_X(x)}{|x|^\alpha} < \infty\] for \(\alpha > 0\), then \[\mathbb{E}[X^{-1}] \text{ exists}.\]
\end{corollary}

Most common distribution functions do not adhere to this corollary, however, the Gamma function does (see example \ref{p:negative}).

If such a moment of negative order exists, we should be able to obtain it using the MGF. In the 20-th century, \citet{cressie1981} have published the following remarkable theorem:

\begin{theorem}\label{t: negative}
    Assuming the negative \(n\)-th raw moment exists, the negative \(n\)-th raw moment can be computed as follows: 
    \[\mathbb{E}[X^{-n}] = \frac{1}{\Gamma(n)} \int_{0}^{\infty} t^{n- 1} M_X(-t) dt\] where \(n\) is a positive integer.
\end{theorem}
The proof of this Theorem can be found in Appendix \ref{pf:MGF}.

Since this is an extension on the regular functions of the MGF, this technique is of interest for this thesis. Thus, it will be shortly be discussed. We compute the first inverse moment of the Gamma distribution, by making use of the latter theorem for the MGF.

\begin{example}
    Let \[f_X(x) \sim \Gamma(\alpha, \lambda) = 
    \frac{x^{\alpha -1} e^{-\lambda x} \lambda^\alpha} {\Gamma(\alpha)}, M_X(t) = \left(\frac{\lambda}{\lambda - t}\right)^\alpha\]
    \[\mathbb{E}[X^{-1}] = \frac{1}{\Gamma(1)} \int_{0}^{\infty} t^{( 1 - 1)} \left(\frac{\lambda}{\lambda - (-t)}\right)^\alpha dt =  \int_{0}^{\infty} \left(\frac{\lambda}{\lambda + t}\right)^\alpha dt\]
    \[ = \lambda^\alpha \int_{0}^{\infty} (\lambda + t)^{-\alpha} dt, \text{ Let } u = \lambda + t, \frac{du}{dt} = 1, dt = du:\]
    \[ \lambda^\alpha \int_{0}^{\infty} u^{-\alpha} du
    =  \lambda^\alpha \frac{u^{ 1-\alpha}}{1 -\alpha}\Big|_{0}^{\infty} = \lambda^\alpha \frac{(\lambda + t)^{1 -\alpha}}{1 -\alpha}\Big|_{0}^{\infty}\]
    \[= \lambda^\alpha\left( 0 - \frac{\lambda^{ 1 - \alpha}}{1 -\alpha}\right) = \frac{-\lambda}{ 1 - \alpha} = \frac{\lambda}{\alpha - 1}.\] Which corresponds with our result from example \ref{p:negative}.
\end{example}

\subsubsection{Extending the Moment Generating Function to fractional order}
Now that we have introduced the definitions of the MGF and discussed a number of relevant properties, we will combine these with the techniques developed in section \ref{s:calculus}. Therefore, we can at last obtain moments of fractional order using the MGF.


To avoid confusion regarding what fractional derivative is being used in combination with the MGF, we will from now on, work with the following notation:
\begin{definition}\label{d: MGF}
    We define the MGF of order \(\alpha \in \mathbb{R}\) by \(\leftindex_{RL}{M}_X^{(\alpha)}, \leftindex_{CF}{M}_X^{(\alpha)}, \leftindex_{GL}{M}_X^{(\alpha)}\) for the MGF in combination with the Riemann-Liouville, Caputo-Fabrizio and Grünwald-Letnikov fractional derivative respectively.
\end{definition}
\begin{remark}
    The three properties mentioned in proposition \ref{p: moments} still hold for the MGF of fractional order. This is the case as the first property makes makes use of the derivative of order 0. Which has been defined to just be the original function itself as stated in the second property of proposition\ref{p: calculus}. The other two properties do not involve any derivatives of any order. Thus, they are generally applicable to the MGF, regardless of its order or kind of derivative.
\end{remark}

Before stating any results about the accuracy of these new MGF expressions, we first consider an explicit example to illustrate the interaction between fractional derivatives and the MGF. 

\begin{example}
    \begin{enumerate}[(i)]
        \item We let \(f_X(x_i) \sim Bernoulli(p)\), with \(\mathbb{P}(X = 1) = p\) and with associated MGF expression: \(M_X(t) = (1 - p) + p \cdot\exp(t)\), now we consider \[\leftindex_{CF}{M}_X^{(\frac{1}{2})} = \frac{1}{1 - \frac{1}{2}}  \int_{-h}^{t} \exp\left(\frac{-\frac{1}{2}}{1 - \frac{1}{2}}(t-s)\right) M_X'(s) ds.\] In this case, the domain of \(M_X(t) = (-\infty, \infty)\), so we let \(-h = -\infty\), and \(M_X'(s) = p\cdot \exp(s)\), thus we obtain:
        \[\leftindex_{CF}{M}_X^{(\frac{1}{2})} = 2  \int_{-\infty}^{t} \exp\left((s-t)\right) \cdot (p\cdot \exp(s)) ds.\]
        \[= 2p \cdot \exp(-t) \int_{-\infty}^{t}\exp(2s) ds\] 
        \[= p\cdot \exp(-t) \left(\exp(2s) \Big|_{-\infty}^{t}\right) = p\cdot \exp(t)\]
        Now, all that is left to do is set \(t = 0\) and we obtain that \(\leftindex_{CF}{M}_X^{(\frac{1}{2})} = p\).
        \item Computing \(\mathbb{E}[{X^{\frac{1}{2}}}]\) in the traditional fashion, we obtain: 
        \[\mathbb{E}[{X^{\frac{1}{2}}}] = \sum_x x^{\frac{1}{2}} \mathbb{P}(X = x) = 0^{\frac{1}{2}} \cdot \mathbb{P}(X = 0) + 1^{\frac{1}{2}} \cdot \mathbb{P}(X = 1)\]
        \[ = 0 \cdot(1 - p) + 1 \cdot p = p\]
    \end{enumerate}
    
\end{example}
It is amazing and maybe even somewhat surprising that both expressions obtain the same result. Indeed, this result is actually more of a coincidence. It is important to note that this agreement of results is coincidental and specific to the chosen distribution. Namely, all raw higher moments of a Bernoulli random variable are \(p\). If we had taken any other distribution in combination with a moment of fractional order, it becomes highly likely that the MGF returns a different value compared to the traditional method of computing moments. What is more, if we were to take \[\leftindex_{RL}{M}_X^{(\frac{1}{2})}  = \frac{d}{dt} \frac{1}{\sqrt{\pi}}  \int_{-h}^{t} (t - s)^{\frac{-1}{2}} f(s) ds\] we obtain an integral which may diverge based on the choice of \(-h\) . These observations lead to the following theorem.

\begin{theorem}\label{t: MGF_accurate}
     Consider the three MGF's as defined in definition \ref{d: MGF}. Assume \(\leftindex_{RL}{M}_X^{(\alpha)}\) and \(\leftindex_{GL}{M}_X^{(\alpha)}\) are well defined on some open interval \((-h, h)\), then the MGF expressions \(\leftindex_{RL}{M}_X^{(\alpha)}\) and \(\leftindex_{GL}{M}_X^{(\alpha)}\) accurately obtain raw moments of order \(\alpha \in \mathbb{R}\)
    
\end{theorem}
The proof can be found in Appendix \ref{pf:MGF}.
\newline
Unfortunately, this result does not hold for \(\leftindex_{CF}{M}_X^{(\alpha)}\), which leads to the following theorem.

\begin{theorem}\label{t: MGF_inaccurate}
    Consider the three MGF's as defined in definition \ref{d: MGF}. Assume \(\leftindex_{CF}{M}_X^{(\alpha)}\) is well defined on some open interval \((-h, h)\), then the MGF \(\leftindex_{CF}{M}_X^{(\alpha)}\) inaccurately approximates moments of order \(\alpha \in \mathbb{R}\) with approximation error given by
    \[
\begin{cases} 
    \displaystyle \int_{-\infty}^{\infty} x^\alpha  f_X(x) dx -  \displaystyle \int_{-\infty}^{\infty}  \frac{x^{n+1} }{(1 - \beta)x + \beta} f_X(x) dx & \text{if } X \text{ is continuous,} \\ 
    \displaystyle \sum_{i} \left(x_i^\alpha -  \frac{x_i^{n+1} }{(1 - \beta)x_i + \beta}\right) f_X(x_i) & \text{if } X \text{ is discrete.} 
\end{cases}
\] with \(\alpha \in \mathbb{R}, \beta = \alpha - n \text{ and } n = \lfloor \alpha \rfloor.\)
    
\end{theorem}
The proof can be found in Appendix \ref{pf:MGF}.

\begin{remark}
    In the case when \(\alpha \in \mathbb{N}\), we have that \(n = \alpha\), and thus \( \beta = 0\), therefore, \(\leftindex_{CF}{M}_X^{(\alpha)}\) is accurate for integer orders. This is an expected result, as the MGF for integer moments is accurate and from section \ref{s:calculus} we know that \(D_{CF}^{\alpha + \beta}f(x) = D_{CF}^\alpha(D_{CF}^\beta f(x))\), with \(\alpha \in \mathbb{N}, \beta \in [0, 1).\) In this case, let \(\beta = 0\). So we get \(D_{CF}^{\alpha + 0}f(x) = D_{CF}^{\alpha}f(x)\) which is just a regular derivative of integer order.
\end{remark}
