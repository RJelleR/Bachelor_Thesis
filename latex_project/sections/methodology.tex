\section{Methodology}\label{s:methodology}
\subsection{Existing methods of obtaining fractional moments}
The formal definition of the statistical moments mentioned in the previous section is as follows: The \(n\)-th moment of a random variable \(X\), with probability density function (PDF) \(f_X(x)\) 
\[
\mathbb{E}[X^n] = 
\begin{cases} 
\int_{-\infty}^{\infty} (x - c)^n f_X(x) \, dx & \text{if } f_X(x) \text{ is continuous,} \\ 
\sum_{i} (x_i - c_ i)^n f_X(x_i) & \text{if } f_X(x) \text{ is discrete.} 
\end{cases}, n \in \mathbb{N}
\]

The traditional method of computing fractional moments is rather straightforward. Similarly to integer moments, one simply computes the summation (in the discrete case) or integral (in the continuous case) of \(x^\alpha \cdot f_x(x)\), where \(f_x(x)\) denotes the probability density function of the random variable \(X\). In the context of fractional moments, \(\alpha \in \mathbb{R}\) instead of \(\mathbb{Z}\) (assuming that negative moments exist). \cite{hansen2024} introduce an alternative approach to computing fractional moments using the complex moment generating function (CMGF), which they apply in the context of the aforementioned GARCH models. One of their key expressions is given by:

\[\mathbb{E}\left| X - \xi \right|^\alpha = \frac{\Gamma(\alpha+1)}{2\pi} \int_{-\infty}^{\infty} \frac{e^{-\xi z} M_X(z) + e^{\xi z} M_X(-z)}{z^{\alpha+1}} dt\] \(\text{ where } z = s + it, s \in \mathbb{N_+}, \xi \in \mathbb{R} \) and \(\alpha\) of course the order of the moment.

This formulation extends upon the traditional moment generating function (MGF) but avoids the process of taking derivatives, making it computationally efficient. The inclusion of the Gamma function is logical, as it extends the factorial function to real values, aligning well with the computation of fractional moments. Since this method relies on integration, rather than differentiation, it avoids numerical issues that might arise when computing derivatives, such as obtaining rather great approximation errors.

\subsection{Obtaining fractional moments by using the moment generating function}
While the CMGF method provides an efficient  and elegant alternative to the traditional method, this thesis explores a different approach: computing fractional moments directly by applying fractional derivatives to the MGF. The MGF is widely used for computing integer moments by differentiation around zero. Extending this approach to fractional orders requires us to take fractional derivatives. Thus, we need to define such fractional derivative operators. These fractional derivatives have a long history and often make use of the aforementioned Gamma function in combination with some integral. This means that, for continuous random variables, where we integrate the MGF, we will have to do double integration. A consequence might be that obtaining analytical expressions of these moments may not always be possible. A lot of alternative expressions of these fractional derivatives have been created, mostly based on different interpretations of the latter in the field of physics. In this thesis, we will focus on computing the MGF using the Riemann-Liouville derivative, the Caputo-Fabrizio derivative and the Grünwald-Letnikov fractional derivative. Each of these fractional derivatives in combination with the MGF might lead to different moments expressions for the same distribution and same fractional order of the moment. Thus, it is essential to compare each of these definitions with the traditional way of computing fractional moments, to derive their accuracy and conclude which approach is most suitable for fractional moment computation. Similar to the expression of the moments of a random variable, their errors may also be hard to derive analytically, depending on its distribution.

\subsection{Order and methods of research}
To realize the differences between the aforementioned fractional derivatives, the order of research will be as follows. First, a mathematical groundwork for the fractional derivatives will be laid, in which a number of their respective properties will be discussed. Next, we will revise some basic definitions of statistical moments and the moment generating function and see how some of these properties change when considering fractional moments in combination with the moment generating function. If possible, analytical expressions of the errors for each fractional derivative in combination with the MGF will be computed, which will conclude the theoretical research of this thesis. We will conduct a simulation using the programming language Julia, to now obtain numerical errors instead of analytical errors. The simulation will evaluate each MGF-based method using statistical measures, including minimum, maximum, and average errors. Finally, we will consider a practical case (WIP: TO BE DECIDED) in which all three derivatives will once again be compared.