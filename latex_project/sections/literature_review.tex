\section{Literature Review}
\subsection{Application of fractional moments}
Fractional moments play a significant role in a variety fields, including finance, economics, and statistics An example of the latter is its application in approximating integer moments as described by \cite{inverardi2024}. This is especially useful when (for \(\alpha \in \mathbb{R}\)) the \(\lceil\alpha \rceil\)-th central integer moment might not exist, while its fractional moment does. Finding an existing fractional moment close to a non-existing integer moment, could still provide one with information about this integer moment. For example, the student-t distribution with \(\nu = 2\) degrees of freedom only has central and raw moments of order \(k\), where \( 0 < k < \nu\). This implies that its second central moment \((k = 2)\) does not exist. One could however consider the \(k-th\) central moment where \(k = 1.95\) and interpret its value as the variance of the distribution. Fractional moments are also used in financial modelling, particularly in the context of Generalized Autoregressive Conditional Heteroskedasticity (GARCH) models. These models are commonly applied to time series data such as financial returns and capture the dynamic volatility that changes over time. The GARCH model achieves this by modeling the volatility based on the returns and variances of previous time periods. \cite{hansen2024} have obtained a method of finding fractional absolute moments of the cumulative return, which would have been impossible when using any other method. \cite{gyzl2013} have also introduced the usage of Fractional moments in risk-models, specifically insurance models. In such models, often the probability density function of total ruin, the event where an insurance company's capital becomes negative, is unknown. The so-called "Method of maximum entropy" has been developed to find these densities. This method takes fractional moments as its input, as they have been proven to be able to characterize its distribution \cite{lin1992}. This method has proven to be a useful alternative to existing methods, such as inverse Laplace transformations. This is the case as this new method takes less values than the latter as input, making it computationally more efficient. Beyond finance and risk modelling, fractional moments also have important applications in engineering. Examples include optimizing signal processing and control systems as well as studying the response characteristics of random vibration systems. \cite{wang2025} has shown that when using the concept of fractional moments for the latter, accuracy and stability is higher compared to traditional methods, such as Taylor expansions. Furthermore, in terms of simplicity and efficiency, the method of fractional moments is advantageous, as its computation steps are straightforward and avoid convergence issues, significantly reducing the resources required for computation. Working with fractional moments has allowed \cite{wang2025} to obtain both analytical, as well as numerical solutions to problems within their research field, which again proves its viability. Another application within the field of engineering, can be found in the identification of distributions of non-linear systems. \cite{dimatteo2014} have shown that complex fractional moments allow one to solve equations such as the Kolmogorov or Fokker-Planck equation, a characterization of continuous-time Markov processes. After performing a Mellin transformation on this system of equations, the resulting system is a linear system in terms of complex fractional moments. The latter can now be solved rather easily and taking the Inverse Mellin transformation on these solutions immediately provides one with the solutions of the non-linear system. Advantages of using complex fractional moments instead of integer moments is that, when applying the Mellin transformation, the relevant PDF is restored on its entire support. This is not the case for the integer moments. This method of using complex fractional moments has been proven to have a rather high accuracy and is applicable to any stable kind of non-linear system of equations \cite{dimatteo2014}.

\subsection{Obtaining fractional moments}
The traditional method of computing fractional moments is rather straightforward. Similarly to integer moments, one simply computes the summation (in the discrete case) or integral (in the continuous case) of \(x^\alpha \cdot f_x(x)\), where \(f_x(x)\) denotes the probability density function of the random variable \(X\). In the context of fractional moments, \(\alpha \in \mathbb{R}\) instead of \(\mathbb{Z}\) (assuming that negative moments exist). \cite{hansen2024} introduce an alternative approach to computing fractional moments using the complex moment generating function (CMGF), which they apply in the context of the aforementioned GARCH models. One of their key expressions is given by:

\[\mathbb{E}\left| X - \xi \right|^r = \frac{\Gamma(r+1)}{2\pi} \int_{-\infty}^{\infty} \frac{e^{-\xi z} M_X(z) + e^{\xi z} M_X(-z)}{z^{r+1}} dt\] \(\text{ where } z = s + it, s \in \mathbb{N_+}, \xi \in \mathbb{R} \) and \(r\) of course the order of the moment.

This formulation extends upon the traditional moment generating function (MGF) but avoids the process of taking derivatives, making it computationally efficient. The inclusion of the Gamma function is logical, as it extends the factorial function to real values, aligning well with the computation of fractional moments. Since this method relies on integration, rather than differentiation, it avoids numerical issues that might arise when computing derivatives, such as obtaining rather great approximation errors.
While the CMGF method provides an efficient  and elegant alternative, this thesis explores a different approach: computing fractional moments directly by applying fractional derivatives to the MGF. The MGF is widely used for computing integer moments by differentiation around zero. To extend this approach to fractional orders requires us to take fractional derivatives. Thus, we need to define such fractional derivative operators. These fractional derivatives have a long history and often make use of the aforementioned Gamma function in combination with some integral. This means that, for continuous random variables, where we integrate the MGF, we will have to do double integration. A consequence might be that obtaining analytical expressions of these moments may not always be possible. A lot of alternative expressions of these fractional derivatives have been created, mostly based on different interpretations of the latter in the field of physics. This implies that different expressions of fractional derivatives in combination with the MGF might obtain different  moments expressions for the same distribution and same fractional order of the moment. Thus, it is essential to compare each of these definitions with the traditional way of computing fractional moments, to derive their accuracy and conclude which approach is most suitable for fractional moment computation. Similar to the expression of the moments of a random variable, their 'Biases' may also be hard to derive analytically, depending on its distribution.



\subsection{Fractional derivatives}
Although not the main topic of interest of this thesis, it is useful to have some knowledge about the history and applications of fractional derivatives. The study of fractional derivatives has been relevant as early as the year 1695 when the concept of such a derivative was implicitly discussed by Leibniz and Bernoulli \cite{katugampola2014}. Since then, numerous definitions of fractional derivatives have been developed. The best-known definition of the fractional derivative is the Riemann-Liouville derivative, its upper derivative of a function \(f(x)\) of order \(\alpha\) is denoted as 
\[\frac{d^n}{dx^n}\frac{1}{\Gamma(n -\alpha)} \int_{a}^{x} (x-t)^{n - \alpha -1} f(t) dt, \text{ where }n = \lceil\alpha \rceil \] \cite{kilbas2006}.
Michele \cite{caputo1967} defined a variation on this derivative, where instead of \(\frac{d^n}{dx^n}\) in front of the integral, we have \(\frac{d^n}{dt^n}\) inside the integral. Due to this adjustment, it is possible to have initial value conditions expressed as the traditional derivatives of integer-order, which made these fractional differential equation problems more intuitive. Other fractional derivatives, such as the \cite{hadamard1892} and \cite{riesz1949} derivative, have been defined to take advantage of particular beneficial properties. For example, each of the latter derivatives can be written as a Fourier transformation. As a consequence, the analytical expressions, can often be simplified. A rather unique derivative is the Grünwald-Letnikov derivative which, in contrast to all the aforementioned derivatives, is not based on integral. Instead, it generalizes the difference quotient, \(\frac{f(x+h) - f(x)}{h}\),  to fractional orders using binomial coefficients \cite{atici2021}. This variety of definitions emphasises how dependent fractional derivatives are  on different physical interpretations and practical applications.
Beyond their theoretical significance, fractional derivatives have been of significant importance in various scientific fields since the 19th century. Examples include fractional Fourier transformations, a generalization of the regular Fourier transformations \cite{missbauer2012}, fractional diffusion equation models, describing the motion of particles in liquids as a consequence of thermal molecular motions \cite{einstein1905} and the fractional Schrödinger equation, a generalization of the Schrödinger equation, often used in quantum mechanics \cite{laskin2002}. Their applications are less common in the fields of finance or economics, as fractional derivatives are mainly used to describe natural phenomena \cite{boulaaras2023}. Yet they still offer some great potential. (Symmetric) Levy flights make use of fractional derivatives in order to solve partial differential equations which describe random walk processes in time series \cite{scalas2000}. The development of fractional derivatives also led to the notion of fractional Brownian motions, a generalization of the Brownian motion \cite{mandelbrot1968}. The latter is a continuous-time stochastic process which, similar to Levy flights, may be used to model random walk processes.