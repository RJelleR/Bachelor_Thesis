\section*{Abstract}
This thesis explores a novel approach for computing moments of fractional order by combining the Moment Generating Function with various types of fractional derivatives. The study demonstrated that the Moment Generating Function in combination with the Riemann-Liouville or Grünwald-Letnikov derivative obtains accurate values of moments of fractional orders. In contrast, the Moment Generating Function in combination with the Caputo-Fabrizio derivative consistently underestimates these moments, particularly as the distribution parameters or forecasting horizons increase.
By making use of moments of fractional order, statistics such as the standard deviation, skewness and kurtosis were computed to analyze and forecast volatility of stock returns in financial markets.
The concept of Lower Partial Moments was extended to fractional orders, providing insights into the frequency and magnitude of downside risk. Additionally, moments of fractional order were implemented into an observation-driven regression model, reducing errors and improving prediction performance compared to traditional models. Numerical integration was required due to the absence of closed-form expressions of the Moment Generating Function. As a result, absolute moments were employed to ensure numerical stability. This solution comes at the cost of interpretability of the results, a problem that often occurs when considering moments of fractional order.

\bigskip\noindent
\textbf{Keywords:} fractional moments, moment generating function, fractional calculus, Caputo-Fabrizio, volatility forecasting, observation-driven regression models