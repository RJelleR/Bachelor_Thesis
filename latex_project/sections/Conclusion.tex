\section{Conclusion}\label{s:con}
This paper has explored a relatively new method for computing moments of fractional order. This novel approach included combining Moment Generating Functions with various different fractional derivatives. It was shown that the Moment Generating Function in combination with either the Riemann-Liouville fractional derivative or the Grünwald-Letnikov fractional derivative lead to accurate computations of fractional moments. In contrast, the Moment Generating Function in combination with the Caputo-Fabrizio fractional derivative leads to systematically underestimation of moments of fractional order.
\newline
It was shown that in general the mean and maximum error, the standard deviation, skewness and the coefficient of variation of the error increase as the parameter values of the underlying distribution grow. Nevertheless, the Caputo-Fabrizio Moment Generating Function remains accurate for integer orders and in general, most of the errors are clustered around zero. 
\newline
We demonstrated the practical relevance of moments of fractional order through the application of volatility forecasting in financial markets. These moments were used to derive informative statistics for future stock returns and volatility. The values obtained using the Riemann-Liouville and Grünwald-Letnikov Moment Generating Functions aligned closely with the empirical values, thus confirming their effectiveness in real world applications.
\newline
Additionally, we extended the concept of Lower Partial Moment to fractional orders, which represent the frequency and magnitude of downside deviations. In both applications, the Caputo-Fabrizio Moment Generating Function was shown to underestimate the moments of fractional order. In particular when the forecasting horizon increased. This suggests that the Caputo-Fabrizio Moment Generating Function is not well suited for financial risk analysis, as it tends to understate volatility and associated risk. 
\newline
Moreover, moments of fractional order were implemented as regressors in an observation-driven regression model. This approach allowed for a more flexible \(\beta_t\) compared to the \(\beta_t\) of the standard model, which led to reduced errors and improved forecasting accuracy compared to a standard model. In the absence of closed-form expressions of the Moment Generating Function in this setting, it was required to make use of traditional numerical integration. To ensure numerical stability we had the option to either employ absolute or complex moments. While these options are effective, they somewhat limit the interpretability of the results, compared to unrestricted fractional moments. This limitation could be avoided if closed form expressions of the Moment Generating Function are available.
\newline

Future research could include exploring Moment Generating Functions in combinations with different fractional derivatives as those proposed by \cite{hadamard1892}, \cite{riesz1949} and \cite{marchaud1927}. It would be valuable to assess which combinations obtain accurate values of moments of fractional order and whether a general theorem regarding the accuracy of the Moment Generating Function in combination with fractional derivatives can be established. Another topic of interest would be to consider computing fractional moments of multivariate distributions as briefly mentioned in \cite{hansen2024}. This would allow us to compute fractional cross-moments to explore dependencies between different random variables. On a practical level, developing Julia packages capable of performing symbolic differentiation similar to SymPy in Python \cite{meurer2017} would reduce reliance on manual derivation. Currently, the lack of such tools in Julia requires all derivative expressions to be computed by hand before use.